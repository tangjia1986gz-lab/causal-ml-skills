%%%%%%%%%%%%%%%%%%%%%%%%%%%%%%%%%%%%%%%%%%%%%%%%%%%%%%%%%%%%%%%%%%%%%
%% Economics Research Paper Template (AER Style)
%%
%% Usage: Replace [BRACKETED] content with your text
%% Compile with: pdflatex -> bibtex -> pdflatex -> pdflatex
%%%%%%%%%%%%%%%%%%%%%%%%%%%%%%%%%%%%%%%%%%%%%%%%%%%%%%%%%%%%%%%%%%%%%

\documentclass[12pt]{article}

%% ===== Packages =====
\usepackage[utf8]{inputenc}
\usepackage[T1]{fontenc}
\usepackage{amsmath,amssymb,amsthm}
\usepackage{graphicx}
\usepackage{booktabs}           % Professional tables
\usepackage{threeparttable}     % Table notes
\usepackage{natbib}             % Author-year citations
\usepackage{hyperref}
\usepackage{setspace}
\usepackage{geometry}
\usepackage{float}
\usepackage{subcaption}
\usepackage{appendix}
\usepackage{xcolor}

%% ===== Page Setup =====
\geometry{
    letterpaper,
    left=1in,
    right=1in,
    top=1in,
    bottom=1in
}

%% ===== Spacing =====
\doublespacing
\setlength{\parskip}{0pt}
\setlength{\parindent}{0.5in}

%% ===== Citation Style =====
\bibliographystyle{aer}
% Alternative: \bibliographystyle{chicago}

%% ===== Custom Commands =====
\newcommand{\E}{\mathbb{E}}
\newcommand{\Var}{\text{Var}}
\newcommand{\Cov}{\text{Cov}}
\newcommand{\plim}{\text{plim}}

%% ===== Table/Figure Notes Environment =====
\newenvironment{tablenotes}{\footnotesize\par\medskip}{\par}

%% ===== Title Information =====
\title{[TITLE: Clear, Specific, Informative]\\[1ex]
\large [Subtitle if needed]}

\author{
    [First Author]\thanks{[Affiliation]. Email: [email].
    [Acknowledgments can go here or in separate section.]}
    \and
    [Second Author]\thanks{[Affiliation]. Email: [email].}
}

\date{[Month Year]\\[1ex]
\small Preliminary and Incomplete}

%%%%%%%%%%%%%%%%%%%%%%%%%%%%%%%%%%%%%%%%%%%%%%%%%%%%%%%%%%%%%%%%%%%%%
\begin{document}
%%%%%%%%%%%%%%%%%%%%%%%%%%%%%%%%%%%%%%%%%%%%%%%%%%%%%%%%%%%%%%%%%%%%%

\maketitle

%% ===== Abstract =====
\begin{abstract}
\noindent
[CONTEXT: 1-2 sentences on why this matters.]
[GAP: What's missing from existing research?]
[THIS PAPER: What do you do and how?]
[RESULTS: Main finding with specific numbers.]
[IMPLICATION: Why should readers care?]

\bigskip
\noindent
\textbf{JEL Codes:} [C21, J31, etc.]

\noindent
\textbf{Keywords:} [keyword1], [keyword2], [keyword3]
\end{abstract}

\thispagestyle{empty}
\newpage
\setcounter{page}{1}

%%%%%%%%%%%%%%%%%%%%%%%%%%%%%%%%%%%%%%%%%%%%%%%%%%%%%%%%%%%%%%%%%%%%%
%% INTRODUCTION
%%%%%%%%%%%%%%%%%%%%%%%%%%%%%%%%%%%%%%%%%%%%%%%%%%%%%%%%%%%%%%%%%%%%%
\section{Introduction}
\label{sec:intro}

% Paragraph 1: Motivation
[Open with the big question or a striking fact that hooks the reader. Why does this matter for policy, theory, or practice? What's at stake?]

% Paragraph 2: This paper
[What specifically do you do? What is your identification strategy? One sentence on data, one on method.]

% Paragraph 3: Results preview
[State your main quantitative finding. Include specific numbers with standard errors or confidence intervals. Give magnitude context.]

% Paragraph 4: Contribution
[How does this advance the literature? What do you do that others haven't? Be specific about your value-add. Consider: ``Our contribution is threefold. First,... Second,... Third,...'']

% Roadmap paragraph
The remainder of this paper is organized as follows. Section \ref{sec:background} describes the institutional background and data. Section \ref{sec:methods} presents our empirical strategy. Section \ref{sec:results} reports results. Section \ref{sec:conclusion} concludes.

%%%%%%%%%%%%%%%%%%%%%%%%%%%%%%%%%%%%%%%%%%%%%%%%%%%%%%%%%%%%%%%%%%%%%
%% INSTITUTIONAL BACKGROUND AND DATA
%%%%%%%%%%%%%%%%%%%%%%%%%%%%%%%%%%%%%%%%%%%%%%%%%%%%%%%%%%%%%%%%%%%%%
\section{Background and Data}
\label{sec:background}

\subsection{Institutional Setting}
\label{subsec:setting}

[Describe the policy, institution, or setting. Why is this useful for identification? What features does the reader need to understand?]

\subsection{Data}
\label{subsec:data}

[Data source, coverage, sample construction. Key variable definitions. Any data limitations and how you address them.]

% Summary statistics table
\begin{table}[htbp]
\centering
\caption{Summary Statistics}
\label{tab:sumstats}
\begin{threeparttable}
\begin{tabular}{lccccc}
\toprule
 & Mean & SD & Min & Max & N \\
\midrule
\multicolumn{6}{l}{\textit{Panel A: Outcome Variables}} \\
[Outcome 1] & [X.XX] & ([X.XX]) & [X.XX] & [X.XX] & [N] \\
[Outcome 2] & [X.XX] & ([X.XX]) & [X.XX] & [X.XX] & [N] \\
& & & & & \\
\multicolumn{6}{l}{\textit{Panel B: Treatment and Controls}} \\
[Treatment] & [X.XX] & ([X.XX]) & [X.XX] & [X.XX] & [N] \\
[Control 1] & [X.XX] & ([X.XX]) & [X.XX] & [X.XX] & [N] \\
[Control 2] & [X.XX] & ([X.XX]) & [X.XX] & [X.XX] & [N] \\
\bottomrule
\end{tabular}
\begin{tablenotes}
Notes: Standard deviations in parentheses. Sample restricted to [sample definition]. Data source: [source].
\end{tablenotes}
\end{threeparttable}
\end{table}

%%%%%%%%%%%%%%%%%%%%%%%%%%%%%%%%%%%%%%%%%%%%%%%%%%%%%%%%%%%%%%%%%%%%%
%% EMPIRICAL STRATEGY
%%%%%%%%%%%%%%%%%%%%%%%%%%%%%%%%%%%%%%%%%%%%%%%%%%%%%%%%%%%%%%%%%%%%%
\section{Empirical Strategy}
\label{sec:methods}

\subsection{Identification}
\label{subsec:identification}

[What variation do you exploit? Why is this variation plausibly exogenous?]

Our estimating equation is:
\begin{equation}
\label{eq:main}
Y_{it} = \alpha + \beta D_{it} + X'_{it}\gamma + \mu_i + \lambda_t + \varepsilon_{it}
\end{equation}

\noindent where $Y_{it}$ is [outcome] for individual $i$ in period $t$, $D_{it}$ is [treatment indicator], $X_{it}$ is a vector of controls including [list], $\mu_i$ are [individual/firm] fixed effects, $\lambda_t$ are [year/period] fixed effects, and $\varepsilon_{it}$ is the error term.

The parameter of interest is $\beta$, which captures [interpretation under identifying assumption].

\subsection{Identifying Assumptions}
\label{subsec:assumptions}

[State your key identifying assumption clearly. What must be true for $\beta$ to have a causal interpretation?]

\subsection{Threats to Identification}
\label{subsec:threats}

[What could bias your estimates? Address selection, reverse causality, omitted variables. Be proactive.]

%%%%%%%%%%%%%%%%%%%%%%%%%%%%%%%%%%%%%%%%%%%%%%%%%%%%%%%%%%%%%%%%%%%%%
%% RESULTS
%%%%%%%%%%%%%%%%%%%%%%%%%%%%%%%%%%%%%%%%%%%%%%%%%%%%%%%%%%%%%%%%%%%%%
\section{Results}
\label{sec:results}

\subsection{Main Results}
\label{subsec:main_results}

[Walk through your main table. Interpret the economic magnitude, not just statistical significance.]

% Main results table
\begin{table}[htbp]
\centering
\caption{[Descriptive Title: Effect of X on Y]}
\label{tab:main}
\begin{threeparttable}
\begin{tabular}{lccc}
\toprule
 & (1) & (2) & (3) \\
 & [Spec 1] & [Spec 2] & [Spec 3] \\
\midrule
[Treatment Variable] & [X.XXX]*** & [X.XXX]*** & [X.XXX]*** \\
 & ([X.XXX]) & ([X.XXX]) & ([X.XXX]) \\
 & & & \\
[Control 1] & & [X.XXX]** & [X.XXX]** \\
 & & ([X.XXX]) & ([X.XXX]) \\
\midrule
Controls & No & Yes & Yes \\
[Type] FE & No & No & Yes \\
Observations & [N] & [N] & [N] \\
R-squared & [0.XXX] & [0.XXX] & [0.XXX] \\
\bottomrule
\end{tabular}
\begin{tablenotes}
Notes: Standard errors clustered at [level] in parentheses. *** p$<$0.01, ** p$<$0.05, * p$<$0.1. Dependent variable is [outcome]. [Additional notes about specification.]
\end{tablenotes}
\end{threeparttable}
\end{table}

\subsection{Robustness}
\label{subsec:robustness}

[Alternative specifications, placebo tests, sensitivity to key choices. Can be brief if results are in appendix.]

\subsection{Heterogeneity}
\label{subsec:heterogeneity}

[Subgroup analysis if applicable. What drives the effect?]

%%%%%%%%%%%%%%%%%%%%%%%%%%%%%%%%%%%%%%%%%%%%%%%%%%%%%%%%%%%%%%%%%%%%%
%% CONCLUSION
%%%%%%%%%%%%%%%%%%%%%%%%%%%%%%%%%%%%%%%%%%%%%%%%%%%%%%%%%%%%%%%%%%%%%
\section{Conclusion}
\label{sec:conclusion}

[Paragraph 1: Summarize the question and main finding.]

[Paragraph 2: Policy/theoretical implications.]

[Paragraph 3: Limitations and future research. Be honest but don't undermine your findings.]

%%%%%%%%%%%%%%%%%%%%%%%%%%%%%%%%%%%%%%%%%%%%%%%%%%%%%%%%%%%%%%%%%%%%%
%% REFERENCES
%%%%%%%%%%%%%%%%%%%%%%%%%%%%%%%%%%%%%%%%%%%%%%%%%%%%%%%%%%%%%%%%%%%%%
\newpage
\bibliography{references}

%%%%%%%%%%%%%%%%%%%%%%%%%%%%%%%%%%%%%%%%%%%%%%%%%%%%%%%%%%%%%%%%%%%%%
%% APPENDIX
%%%%%%%%%%%%%%%%%%%%%%%%%%%%%%%%%%%%%%%%%%%%%%%%%%%%%%%%%%%%%%%%%%%%%
\newpage
\appendix

\section{Additional Results}
\label{app:additional}

[Additional tables, robustness checks, details on data construction, etc.]

\section{Proofs and Derivations}
\label{app:proofs}

[Mathematical details if applicable.]

%%%%%%%%%%%%%%%%%%%%%%%%%%%%%%%%%%%%%%%%%%%%%%%%%%%%%%%%%%%%%%%%%%%%%
\end{document}
%%%%%%%%%%%%%%%%%%%%%%%%%%%%%%%%%%%%%%%%%%%%%%%%%%%%%%%%%%%%%%%%%%%%%
