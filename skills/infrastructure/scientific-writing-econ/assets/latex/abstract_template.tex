%%%%%%%%%%%%%%%%%%%%%%%%%%%%%%%%%%%%%%%%%%%%%%%%%%%%%%%%%%%%%%%%%%%%%
%% Abstract Templates for Economics Papers
%%
%% Three templates provided:
%% 1. Short abstract (100 words) - for JPE, Econometrica
%% 2. Standard abstract (150 words) - for AER, QJE
%% 3. Extended abstract (200 words) - for working papers
%%%%%%%%%%%%%%%%%%%%%%%%%%%%%%%%%%%%%%%%%%%%%%%%%%%%%%%%%%%%%%%%%%%%%

%%%%%%%%%%%%%%%%%%%%%%%%%%%%%%%%%%%%%%%%%%%%%%%%%%%%%%%%%%%%%%%%%%%%%
%% TEMPLATE 1: SHORT ABSTRACT (100 words)
%% Use for: JPE, Econometrica, very tight word limits
%%%%%%%%%%%%%%%%%%%%%%%%%%%%%%%%%%%%%%%%%%%%%%%%%%%%%%%%%%%%%%%%%%%%%

\begin{abstract}
\noindent
% Context + Question (1 sentence)
Does [X] cause [Y]?

% Method (1 sentence)
We exploit [exogenous variation] using [data] to identify [effect].

% Main result (2 sentences)
[X] increases [Y] by [Z] percent (SE = [#]).
Effects are [largest for / concentrated among] [subgroup].

% Implication (1 sentence)
These findings suggest [policy/theoretical implication].

\medskip
\noindent\textbf{JEL:} [XX, YY] \quad \textbf{Keywords:} [word1, word2, word3]
\end{abstract}

% Word count target: 80-100 words


%%%%%%%%%%%%%%%%%%%%%%%%%%%%%%%%%%%%%%%%%%%%%%%%%%%%%%%%%%%%%%%%%%%%%
%% TEMPLATE 2: STANDARD ABSTRACT (150 words)
%% Use for: AER, QJE, ReStud, most top journals
%%%%%%%%%%%%%%%%%%%%%%%%%%%%%%%%%%%%%%%%%%%%%%%%%%%%%%%%%%%%%%%%%%%%%

\begin{abstract}
\noindent
% Context/Motivation (1-2 sentences)
Understanding [topic] is crucial for [policy/theory] because [reason].
Yet identifying the causal effect of [X] on [Y] is challenging due to [endogeneity concern].

% This paper (1-2 sentences)
This paper exploits [source of exogenous variation] to identify [effect].
We use [data source] covering [sample description] and employ [estimation method].

% Main results (2-3 sentences)
We find that [X] increases [Y] by [Z] percent (SE = [#]).
This effect is [economically significant / modest] --- equivalent to [comparison].
[Secondary finding: heterogeneity, mechanism, or robustness.]

% Implication (1 sentence)
Our findings suggest that [policy implication] and contribute to understanding [broader question].

\medskip
\noindent
\textbf{JEL Codes:} [C21, J31, etc.]

\noindent
\textbf{Keywords:} [keyword1], [keyword2], [keyword3], [keyword4]
\end{abstract}

% Word count target: 130-150 words


%%%%%%%%%%%%%%%%%%%%%%%%%%%%%%%%%%%%%%%%%%%%%%%%%%%%%%%%%%%%%%%%%%%%%
%% TEMPLATE 3: EXTENDED ABSTRACT (200 words)
%% Use for: NBER Working Papers, conference submissions
%%%%%%%%%%%%%%%%%%%%%%%%%%%%%%%%%%%%%%%%%%%%%%%%%%%%%%%%%%%%%%%%%%%%%

\begin{abstract}
\noindent
% Context/Motivation (2 sentences)
[Big-picture question or striking fact that hooks the reader.]
Understanding this is crucial for [policy debates / economic theory / practical decisions] because [specific reason].

% Gap/Challenge (1-2 sentences)
Despite its importance, existing research has not [specific gap].
The key challenge is that [identification problem / data limitation / methodological issue].

% This paper (2 sentences)
We address this challenge by exploiting [source of exogenous variation].
Using [data source] covering [N observations] from [context], we implement [estimation strategy].

% Main results (3-4 sentences)
We find that [X] causes [Y] to increase by [Z] percent (SE = [#]).
To put this in perspective, the effect is equivalent to [meaningful comparison].
[Heterogeneity finding: Effects are largest for / concentrated among X.]
[Robustness: These results are robust to controlling for / alternative specifications.]

% Mechanisms (1 sentence, optional)
[We provide evidence that [mechanism] drives this effect.]

% Implications (1-2 sentences)
These findings have implications for [policy area].
[Broader significance for literature / theory.]

\medskip
\noindent
\textbf{JEL Codes:} [C21, J31, etc.]

\noindent
\textbf{Keywords:} [keyword1], [keyword2], [keyword3], [keyword4], [keyword5]
\end{abstract}

% Word count target: 180-200 words


%%%%%%%%%%%%%%%%%%%%%%%%%%%%%%%%%%%%%%%%%%%%%%%%%%%%%%%%%%%%%%%%%%%%%
%% EXAMPLE: COMPLETED ABSTRACT (Standard length)
%%%%%%%%%%%%%%%%%%%%%%%%%%%%%%%%%%%%%%%%%%%%%%%%%%%%%%%%%%%%%%%%%%%%%

\begin{abstract}
\noindent
Does education cause higher earnings, or do more able individuals simply obtain more schooling?
Understanding this relationship is fundamental for human capital theory and education policy,
yet identifying the causal effect is challenging because ability is unobserved.

We exploit discontinuities in school entry age cutoffs across U.S. states to identify
exogenous variation in years of completed schooling. Using administrative earnings records
linked to birth certificates for 2.3 million individuals, we implement a regression
discontinuity design.

We find that an additional year of education increases earnings by 9.2 percent (SE = 1.4).
This effect---roughly equivalent to the black-white earnings gap---is concentrated among
individuals from disadvantaged backgrounds. Results are robust to bandwidth choice and
alternative specifications.

Our findings suggest that policies expanding educational access can reduce earnings
inequality, with estimated benefit-cost ratios exceeding 3:1.

\medskip
\noindent
\textbf{JEL Codes:} I21, I26, J31

\noindent
\textbf{Keywords:} returns to education, regression discontinuity, human capital, earnings inequality
\end{abstract}

% Word count: 148 words


%%%%%%%%%%%%%%%%%%%%%%%%%%%%%%%%%%%%%%%%%%%%%%%%%%%%%%%%%%%%%%%%%%%%%
%% COMMON OPENING SENTENCES BY TOPIC
%%%%%%%%%%%%%%%%%%%%%%%%%%%%%%%%%%%%%%%%%%%%%%%%%%%%%%%%%%%%%%%%%%%%%

% Labor/Education:
% "Does education cause higher earnings, or do more able individuals simply obtain more schooling?"
% "The minimum wage remains one of the most debated policies in labor economics."
% "Understanding how workers respond to incentives is central to labor market policy."

% Development:
% "Can cash transfers reduce poverty in the long run, or do recipients return to baseline after transfers end?"
% "Over 700 million people live in extreme poverty, yet effective interventions remain debated."
% "Microfinance has been promoted as a solution to poverty, but rigorous evidence is limited."

% Health:
% "Health insurance coverage has expanded dramatically, but effects on health remain unclear."
% "The opioid crisis has claimed over 500,000 lives, yet the role of prescribing remains debated."

% Public Finance:
% "Tax policy affects both revenue and behavior, but behavioral responses are difficult to estimate."
% "Transfers to low-income households are central to social policy, but optimal design is unclear."

% Trade/IO:
% "Globalization has reshaped labor markets, but the distribution of gains and losses is unclear."
% "Market concentration has increased across industries, raising concerns about competition."


%%%%%%%%%%%%%%%%%%%%%%%%%%%%%%%%%%%%%%%%%%%%%%%%%%%%%%%%%%%%%%%%%%%%%
%% COMMON CLOSING SENTENCES
%%%%%%%%%%%%%%%%%%%%%%%%%%%%%%%%%%%%%%%%%%%%%%%%%%%%%%%%%%%%%%%%%%%%%

% Policy implications:
% "Our findings suggest that [policy X] can [achieve goal Y], with estimated benefit-cost ratios of Z."
% "These results inform the ongoing debate over [policy] by providing causal evidence on [effect]."
% "Policymakers should consider [implication] when designing [type of policy]."

% Theoretical contributions:
% "These findings contribute to understanding [mechanism/theory] by providing evidence that [finding]."
% "Our results suggest that [theoretical model] may need to incorporate [element] to match the data."

% Literature contributions:
% "We contribute to the literature on [topic] by providing the first [type of evidence] on [question]."
% "Our identification strategy addresses concerns about [threat] that limit interpretation of prior work."
