% =============================================================================
% heterogeneity_table.tex - CATE Heterogeneity Results Table Template
% =============================================================================
% Style: AER/NBER publication standards for causal machine learning
% Features:
%   - Conditional Average Treatment Effects (CATE) by subgroup
%   - Causal forest variable importance
%   - Best linear projection results
%   - Treatment effect quantiles
%   - Policy targeting metrics
%
% Placeholders (replace with your data):
%   - {{TITLE}}          : Table title
%   - {{LABEL}}          : LaTeX label for cross-referencing
%   - {{CATE_*}}         : Conditional treatment effects
%   - {{SE_*}}           : Standard errors
%   - {{VIMP_*}}         : Variable importance scores
% =============================================================================

\documentclass[12pt]{article}
% =============================================================================
% common_preamble.tex - Shared LaTeX Preamble for Causal Inference Tables/Figures
% =============================================================================
% Usage: % =============================================================================
% common_preamble.tex - Shared LaTeX Preamble for Causal Inference Tables/Figures
% =============================================================================
% Usage: % =============================================================================
% common_preamble.tex - Shared LaTeX Preamble for Causal Inference Tables/Figures
% =============================================================================
% Usage: \input{common_preamble.tex} at the beginning of your document
% Compatible with: pdflatex, xelatex, lualatex
% Style: AER/NBER publication standards
% =============================================================================

% -----------------------------------------------------------------------------
% Required Packages
% -----------------------------------------------------------------------------

% Tables
\usepackage{booktabs}           % Professional table rules (\toprule, \midrule, \bottomrule)
\usepackage{threeparttable}     % Tables with notes
\usepackage{multirow}           % Multi-row cells
\usepackage{array}              % Enhanced column specifications
\usepackage{tabularx}           % Auto-width columns
\usepackage{longtable}          % Multi-page tables

% Number Formatting
\usepackage{siunitx}            % Consistent number formatting and alignment
\sisetup{
    input-symbols = (),         % Allow parentheses in input
    input-open-uncertainty  = , % No uncertainty notation
    input-close-uncertainty = ,
    table-align-text-pre    = false,
    table-align-text-post   = false,
    group-separator         = {,},      % Thousands separator
    group-minimum-digits    = 4,        % Apply grouping for 4+ digits
    round-mode              = places,   % Round to decimal places
    round-precision         = 3,        % Default 3 decimal places
    detect-weight           = true,     % Detect bold
    detect-family           = true,     % Detect font family
}

% Graphics and Plots
\usepackage{graphicx}           % Include graphics
\usepackage{pgfplots}           % Publication-quality plots
\pgfplotsset{compat=1.18}       % Use latest pgfplots features
\usepgfplotslibrary{fillbetween}% For confidence interval shading
\usetikzlibrary{patterns}       % Pattern fills
\usetikzlibrary{arrows.meta}    % Arrow styles

% Colors
\usepackage{xcolor}             % Color support

% Captions
\usepackage{caption}            % Caption customization
\captionsetup{
    font=small,
    labelfont=bf,
    skip=10pt
}

% -----------------------------------------------------------------------------
% Color Definitions (Publication-Friendly Palette)
% -----------------------------------------------------------------------------

% Primary colors for plots
\definecolor{coefblue}{RGB}{31, 119, 180}       % Main coefficient color
\definecolor{ciblue}{RGB}{174, 199, 232}        % Confidence interval fill
\definecolor{treatred}{RGB}{214, 39, 40}        % Treatment/significant
\definecolor{controlgreen}{RGB}{44, 160, 44}    % Control group
\definecolor{refgray}{RGB}{127, 127, 127}       % Reference lines
\definecolor{shadegray}{RGB}{220, 220, 220}     % Shaded regions

% Secondary colors
\definecolor{orange}{RGB}{255, 127, 14}
\definecolor{purple}{RGB}{148, 103, 189}
\definecolor{brown}{RGB}{140, 86, 75}
\definecolor{pink}{RGB}{227, 119, 194}

% -----------------------------------------------------------------------------
% Significance Stars Commands
% -----------------------------------------------------------------------------

% Standard significance stars (AER style)
% Usage: \sym{*}, \sym{**}, \sym{***}
\newcommand{\sym}[1]{\ensuremath{^{#1}}}

% Significance level commands with standard thresholds
% Usage: \stars{0.001} produces ***
\newcommand{\stars}[1]{%
    \ifnum\pdfstrcmp{#1}{0.01}<0 \sym{***}%
    \else\ifnum\pdfstrcmp{#1}{0.05}<0 \sym{**}%
    \else\ifnum\pdfstrcmp{#1}{0.1}<0 \sym{*}%
    \fi\fi\fi%
}

% Pre-defined star commands
\newcommand{\onestar}{\sym{*}}
\newcommand{\twostar}{\sym{**}}
\newcommand{\threestar}{\sym{***}}

% -----------------------------------------------------------------------------
% Table Formatting Commands
% -----------------------------------------------------------------------------

% Standard error in parentheses
% Usage: \se{0.123}
\newcommand{\se}[1]{(#1)}

% Coefficient with standard error
% Usage: \coefse{1.234}{0.123}
\newcommand{\coefse}[2]{#1 \\ \se{#2}}

% Coefficient with SE and stars
% Usage: \coefstars{1.234}{0.123}{**}
\newcommand{\coefstars}[3]{#1\sym{#3} \\ \se{#2}}

% Yes/No indicators for fixed effects
\newcommand{\yes}{Yes}
\newcommand{\no}{No}
\newcommand{\ymark}{\checkmark}

% Panel headers
\newcommand{\panelheader}[1]{\multicolumn{\numcols}{l}{\textit{Panel #1}}}

% -----------------------------------------------------------------------------
% Number Formatting Commands
% -----------------------------------------------------------------------------

% Format large numbers with comma separators
% Usage: \numfmt{12345} produces 12,345
\newcommand{\numfmt}[1]{\num[group-separator={,}]{#1}}

% Format percentages
% Usage: \pct{45.6} produces 45.6%
\newcommand{\pct}[1]{#1\%}

% Format p-values
% Usage: \pval{0.001} produces <0.001 for very small values
\newcommand{\pval}[1]{%
    \ifdim #1 pt < 0.001 pt
        $<$0.001%
    \else
        #1%
    \fi
}

% -----------------------------------------------------------------------------
% Note Formatting for Tables
% -----------------------------------------------------------------------------

% Standard table notes style
\newcommand{\tablenote}[1]{%
    \begin{tablenotes}[flushleft]
        \small
        \item \textit{Notes:} #1
    \end{tablenotes}
}

% Significance note (standard)
\newcommand{\signote}{%
    \sym{*} $p<0.10$, \sym{**} $p<0.05$, \sym{***} $p<0.01$%
}

% Full standard note
\newcommand{\stdnote}[1]{%
    \tablenote{#1 Standard errors in parentheses. \signote.}%
}

% -----------------------------------------------------------------------------
% PGFPlots Style Definitions
% -----------------------------------------------------------------------------

\pgfplotsset{
    % Style for coefficient plots
    coefplot/.style={
        width=0.9\textwidth,
        height=0.6\textwidth,
        xbar,
        bar width=3pt,
        xmin=-2, xmax=2,
        axis lines=left,
        ytick=data,
        enlarge y limits=0.15,
        xlabel={Coefficient Estimate},
        every axis x label/.style={
            at={(ticklabel* cs:1.02)},
            anchor=west,
        },
        x axis line style={-},
        y axis line style={opacity=0},
        tickwidth=0pt,
        clip=false,
    },
    % Style for event study plots
    eventstudy/.style={
        width=0.9\textwidth,
        height=0.5\textwidth,
        axis lines=middle,
        xlabel={Time Relative to Treatment},
        ylabel={Coefficient},
        xmin=-6, xmax=6,
        xtick={-5,-4,-3,-2,-1,0,1,2,3,4,5},
        extra x ticks={0},
        extra x tick style={grid=major, grid style={dashed, red!50}},
        legend pos=north west,
        legend style={draw=none, fill=none},
    },
    % Common style elements
    every axis/.append style={
        font=\small,
        tick label style={font=\small},
        label style={font=\small},
        legend style={font=\small},
    },
}

% -----------------------------------------------------------------------------
% End of Preamble
% -----------------------------------------------------------------------------
 at the beginning of your document
% Compatible with: pdflatex, xelatex, lualatex
% Style: AER/NBER publication standards
% =============================================================================

% -----------------------------------------------------------------------------
% Required Packages
% -----------------------------------------------------------------------------

% Tables
\usepackage{booktabs}           % Professional table rules (\toprule, \midrule, \bottomrule)
\usepackage{threeparttable}     % Tables with notes
\usepackage{multirow}           % Multi-row cells
\usepackage{array}              % Enhanced column specifications
\usepackage{tabularx}           % Auto-width columns
\usepackage{longtable}          % Multi-page tables

% Number Formatting
\usepackage{siunitx}            % Consistent number formatting and alignment
\sisetup{
    input-symbols = (),         % Allow parentheses in input
    input-open-uncertainty  = , % No uncertainty notation
    input-close-uncertainty = ,
    table-align-text-pre    = false,
    table-align-text-post   = false,
    group-separator         = {,},      % Thousands separator
    group-minimum-digits    = 4,        % Apply grouping for 4+ digits
    round-mode              = places,   % Round to decimal places
    round-precision         = 3,        % Default 3 decimal places
    detect-weight           = true,     % Detect bold
    detect-family           = true,     % Detect font family
}

% Graphics and Plots
\usepackage{graphicx}           % Include graphics
\usepackage{pgfplots}           % Publication-quality plots
\pgfplotsset{compat=1.18}       % Use latest pgfplots features
\usepgfplotslibrary{fillbetween}% For confidence interval shading
\usetikzlibrary{patterns}       % Pattern fills
\usetikzlibrary{arrows.meta}    % Arrow styles

% Colors
\usepackage{xcolor}             % Color support

% Captions
\usepackage{caption}            % Caption customization
\captionsetup{
    font=small,
    labelfont=bf,
    skip=10pt
}

% -----------------------------------------------------------------------------
% Color Definitions (Publication-Friendly Palette)
% -----------------------------------------------------------------------------

% Primary colors for plots
\definecolor{coefblue}{RGB}{31, 119, 180}       % Main coefficient color
\definecolor{ciblue}{RGB}{174, 199, 232}        % Confidence interval fill
\definecolor{treatred}{RGB}{214, 39, 40}        % Treatment/significant
\definecolor{controlgreen}{RGB}{44, 160, 44}    % Control group
\definecolor{refgray}{RGB}{127, 127, 127}       % Reference lines
\definecolor{shadegray}{RGB}{220, 220, 220}     % Shaded regions

% Secondary colors
\definecolor{orange}{RGB}{255, 127, 14}
\definecolor{purple}{RGB}{148, 103, 189}
\definecolor{brown}{RGB}{140, 86, 75}
\definecolor{pink}{RGB}{227, 119, 194}

% -----------------------------------------------------------------------------
% Significance Stars Commands
% -----------------------------------------------------------------------------

% Standard significance stars (AER style)
% Usage: \sym{*}, \sym{**}, \sym{***}
\newcommand{\sym}[1]{\ensuremath{^{#1}}}

% Significance level commands with standard thresholds
% Usage: \stars{0.001} produces ***
\newcommand{\stars}[1]{%
    \ifnum\pdfstrcmp{#1}{0.01}<0 \sym{***}%
    \else\ifnum\pdfstrcmp{#1}{0.05}<0 \sym{**}%
    \else\ifnum\pdfstrcmp{#1}{0.1}<0 \sym{*}%
    \fi\fi\fi%
}

% Pre-defined star commands
\newcommand{\onestar}{\sym{*}}
\newcommand{\twostar}{\sym{**}}
\newcommand{\threestar}{\sym{***}}

% -----------------------------------------------------------------------------
% Table Formatting Commands
% -----------------------------------------------------------------------------

% Standard error in parentheses
% Usage: \se{0.123}
\newcommand{\se}[1]{(#1)}

% Coefficient with standard error
% Usage: \coefse{1.234}{0.123}
\newcommand{\coefse}[2]{#1 \\ \se{#2}}

% Coefficient with SE and stars
% Usage: \coefstars{1.234}{0.123}{**}
\newcommand{\coefstars}[3]{#1\sym{#3} \\ \se{#2}}

% Yes/No indicators for fixed effects
\newcommand{\yes}{Yes}
\newcommand{\no}{No}
\newcommand{\ymark}{\checkmark}

% Panel headers
\newcommand{\panelheader}[1]{\multicolumn{\numcols}{l}{\textit{Panel #1}}}

% -----------------------------------------------------------------------------
% Number Formatting Commands
% -----------------------------------------------------------------------------

% Format large numbers with comma separators
% Usage: \numfmt{12345} produces 12,345
\newcommand{\numfmt}[1]{\num[group-separator={,}]{#1}}

% Format percentages
% Usage: \pct{45.6} produces 45.6%
\newcommand{\pct}[1]{#1\%}

% Format p-values
% Usage: \pval{0.001} produces <0.001 for very small values
\newcommand{\pval}[1]{%
    \ifdim #1 pt < 0.001 pt
        $<$0.001%
    \else
        #1%
    \fi
}

% -----------------------------------------------------------------------------
% Note Formatting for Tables
% -----------------------------------------------------------------------------

% Standard table notes style
\newcommand{\tablenote}[1]{%
    \begin{tablenotes}[flushleft]
        \small
        \item \textit{Notes:} #1
    \end{tablenotes}
}

% Significance note (standard)
\newcommand{\signote}{%
    \sym{*} $p<0.10$, \sym{**} $p<0.05$, \sym{***} $p<0.01$%
}

% Full standard note
\newcommand{\stdnote}[1]{%
    \tablenote{#1 Standard errors in parentheses. \signote.}%
}

% -----------------------------------------------------------------------------
% PGFPlots Style Definitions
% -----------------------------------------------------------------------------

\pgfplotsset{
    % Style for coefficient plots
    coefplot/.style={
        width=0.9\textwidth,
        height=0.6\textwidth,
        xbar,
        bar width=3pt,
        xmin=-2, xmax=2,
        axis lines=left,
        ytick=data,
        enlarge y limits=0.15,
        xlabel={Coefficient Estimate},
        every axis x label/.style={
            at={(ticklabel* cs:1.02)},
            anchor=west,
        },
        x axis line style={-},
        y axis line style={opacity=0},
        tickwidth=0pt,
        clip=false,
    },
    % Style for event study plots
    eventstudy/.style={
        width=0.9\textwidth,
        height=0.5\textwidth,
        axis lines=middle,
        xlabel={Time Relative to Treatment},
        ylabel={Coefficient},
        xmin=-6, xmax=6,
        xtick={-5,-4,-3,-2,-1,0,1,2,3,4,5},
        extra x ticks={0},
        extra x tick style={grid=major, grid style={dashed, red!50}},
        legend pos=north west,
        legend style={draw=none, fill=none},
    },
    % Common style elements
    every axis/.append style={
        font=\small,
        tick label style={font=\small},
        label style={font=\small},
        legend style={font=\small},
    },
}

% -----------------------------------------------------------------------------
% End of Preamble
% -----------------------------------------------------------------------------
 at the beginning of your document
% Compatible with: pdflatex, xelatex, lualatex
% Style: AER/NBER publication standards
% =============================================================================

% -----------------------------------------------------------------------------
% Required Packages
% -----------------------------------------------------------------------------

% Tables
\usepackage{booktabs}           % Professional table rules (\toprule, \midrule, \bottomrule)
\usepackage{threeparttable}     % Tables with notes
\usepackage{multirow}           % Multi-row cells
\usepackage{array}              % Enhanced column specifications
\usepackage{tabularx}           % Auto-width columns
\usepackage{longtable}          % Multi-page tables

% Number Formatting
\usepackage{siunitx}            % Consistent number formatting and alignment
\sisetup{
    input-symbols = (),         % Allow parentheses in input
    input-open-uncertainty  = , % No uncertainty notation
    input-close-uncertainty = ,
    table-align-text-pre    = false,
    table-align-text-post   = false,
    group-separator         = {,},      % Thousands separator
    group-minimum-digits    = 4,        % Apply grouping for 4+ digits
    round-mode              = places,   % Round to decimal places
    round-precision         = 3,        % Default 3 decimal places
    detect-weight           = true,     % Detect bold
    detect-family           = true,     % Detect font family
}

% Graphics and Plots
\usepackage{graphicx}           % Include graphics
\usepackage{pgfplots}           % Publication-quality plots
\pgfplotsset{compat=1.18}       % Use latest pgfplots features
\usepgfplotslibrary{fillbetween}% For confidence interval shading
\usetikzlibrary{patterns}       % Pattern fills
\usetikzlibrary{arrows.meta}    % Arrow styles

% Colors
\usepackage{xcolor}             % Color support

% Captions
\usepackage{caption}            % Caption customization
\captionsetup{
    font=small,
    labelfont=bf,
    skip=10pt
}

% -----------------------------------------------------------------------------
% Color Definitions (Publication-Friendly Palette)
% -----------------------------------------------------------------------------

% Primary colors for plots
\definecolor{coefblue}{RGB}{31, 119, 180}       % Main coefficient color
\definecolor{ciblue}{RGB}{174, 199, 232}        % Confidence interval fill
\definecolor{treatred}{RGB}{214, 39, 40}        % Treatment/significant
\definecolor{controlgreen}{RGB}{44, 160, 44}    % Control group
\definecolor{refgray}{RGB}{127, 127, 127}       % Reference lines
\definecolor{shadegray}{RGB}{220, 220, 220}     % Shaded regions

% Secondary colors
\definecolor{orange}{RGB}{255, 127, 14}
\definecolor{purple}{RGB}{148, 103, 189}
\definecolor{brown}{RGB}{140, 86, 75}
\definecolor{pink}{RGB}{227, 119, 194}

% -----------------------------------------------------------------------------
% Significance Stars Commands
% -----------------------------------------------------------------------------

% Standard significance stars (AER style)
% Usage: \sym{*}, \sym{**}, \sym{***}
\newcommand{\sym}[1]{\ensuremath{^{#1}}}

% Significance level commands with standard thresholds
% Usage: \stars{0.001} produces ***
\newcommand{\stars}[1]{%
    \ifnum\pdfstrcmp{#1}{0.01}<0 \sym{***}%
    \else\ifnum\pdfstrcmp{#1}{0.05}<0 \sym{**}%
    \else\ifnum\pdfstrcmp{#1}{0.1}<0 \sym{*}%
    \fi\fi\fi%
}

% Pre-defined star commands
\newcommand{\onestar}{\sym{*}}
\newcommand{\twostar}{\sym{**}}
\newcommand{\threestar}{\sym{***}}

% -----------------------------------------------------------------------------
% Table Formatting Commands
% -----------------------------------------------------------------------------

% Standard error in parentheses
% Usage: \se{0.123}
\newcommand{\se}[1]{(#1)}

% Coefficient with standard error
% Usage: \coefse{1.234}{0.123}
\newcommand{\coefse}[2]{#1 \\ \se{#2}}

% Coefficient with SE and stars
% Usage: \coefstars{1.234}{0.123}{**}
\newcommand{\coefstars}[3]{#1\sym{#3} \\ \se{#2}}

% Yes/No indicators for fixed effects
\newcommand{\yes}{Yes}
\newcommand{\no}{No}
\newcommand{\ymark}{\checkmark}

% Panel headers
\newcommand{\panelheader}[1]{\multicolumn{\numcols}{l}{\textit{Panel #1}}}

% -----------------------------------------------------------------------------
% Number Formatting Commands
% -----------------------------------------------------------------------------

% Format large numbers with comma separators
% Usage: \numfmt{12345} produces 12,345
\newcommand{\numfmt}[1]{\num[group-separator={,}]{#1}}

% Format percentages
% Usage: \pct{45.6} produces 45.6%
\newcommand{\pct}[1]{#1\%}

% Format p-values
% Usage: \pval{0.001} produces <0.001 for very small values
\newcommand{\pval}[1]{%
    \ifdim #1 pt < 0.001 pt
        $<$0.001%
    \else
        #1%
    \fi
}

% -----------------------------------------------------------------------------
% Note Formatting for Tables
% -----------------------------------------------------------------------------

% Standard table notes style
\newcommand{\tablenote}[1]{%
    \begin{tablenotes}[flushleft]
        \small
        \item \textit{Notes:} #1
    \end{tablenotes}
}

% Significance note (standard)
\newcommand{\signote}{%
    \sym{*} $p<0.10$, \sym{**} $p<0.05$, \sym{***} $p<0.01$%
}

% Full standard note
\newcommand{\stdnote}[1]{%
    \tablenote{#1 Standard errors in parentheses. \signote.}%
}

% -----------------------------------------------------------------------------
% PGFPlots Style Definitions
% -----------------------------------------------------------------------------

\pgfplotsset{
    % Style for coefficient plots
    coefplot/.style={
        width=0.9\textwidth,
        height=0.6\textwidth,
        xbar,
        bar width=3pt,
        xmin=-2, xmax=2,
        axis lines=left,
        ytick=data,
        enlarge y limits=0.15,
        xlabel={Coefficient Estimate},
        every axis x label/.style={
            at={(ticklabel* cs:1.02)},
            anchor=west,
        },
        x axis line style={-},
        y axis line style={opacity=0},
        tickwidth=0pt,
        clip=false,
    },
    % Style for event study plots
    eventstudy/.style={
        width=0.9\textwidth,
        height=0.5\textwidth,
        axis lines=middle,
        xlabel={Time Relative to Treatment},
        ylabel={Coefficient},
        xmin=-6, xmax=6,
        xtick={-5,-4,-3,-2,-1,0,1,2,3,4,5},
        extra x ticks={0},
        extra x tick style={grid=major, grid style={dashed, red!50}},
        legend pos=north west,
        legend style={draw=none, fill=none},
    },
    % Common style elements
    every axis/.append style={
        font=\small,
        tick label style={font=\small},
        label style={font=\small},
        legend style={font=\small},
    },
}

% -----------------------------------------------------------------------------
% End of Preamble
% -----------------------------------------------------------------------------


\begin{document}

% =============================================================================
% Template 1: CATE by Subgroups (Causal Forest or DR-Learner)
% =============================================================================

\begin{table}[htbp]
\centering
\caption{{{TITLE}}: Heterogeneous Treatment Effects by Subgroup}
\label{tab:{{LABEL}}_cate}

\begin{threeparttable}
\begin{tabular}{l
    S[table-format=-1.3]    % CATE
    S[table-format=1.3]     % SE
    S[table-format=-1.3]    % CI Lower
    S[table-format=-1.3]    % CI Upper
    S[table-format=5.0]     % N
}
\toprule
Subgroup            & {CATE} & {SE} & {95\% CI Lower} & {95\% CI Upper} & {N} \\
\midrule

\multicolumn{6}{l}{\textit{Panel A: Overall Effect}} \\
\addlinespace[0.3em]
Full Sample         & {{CATE_FULL}}  & {{SE_FULL}}  & {{CI_LO_FULL}}  & {{CI_HI_FULL}}  & {{N_FULL}}  \\
\addlinespace[0.5em]

\multicolumn{6}{l}{\textit{Panel B: By Gender}} \\
\addlinespace[0.3em]
Male                & {{CATE_MALE}}  & {{SE_MALE}}  & {{CI_LO_MALE}}  & {{CI_HI_MALE}}  & {{N_MALE}}  \\
Female              & {{CATE_FEM}}   & {{SE_FEM}}   & {{CI_LO_FEM}}   & {{CI_HI_FEM}}   & {{N_FEM}}   \\
Difference          & {{CATE_GDIFF}} & {{SE_GDIFF}} & {{CI_LO_GDIFF}} & {{CI_HI_GDIFF}} &             \\
\addlinespace[0.5em]

\multicolumn{6}{l}{\textit{Panel C: By Age Group}} \\
\addlinespace[0.3em]
Young ($<$ 35)      & {{CATE_YOUNG}} & {{SE_YOUNG}} & {{CI_LO_YOUNG}} & {{CI_HI_YOUNG}} & {{N_YOUNG}} \\
Middle (35--55)     & {{CATE_MID}}   & {{SE_MID}}   & {{CI_LO_MID}}   & {{CI_HI_MID}}   & {{N_MID}}   \\
Old ($>$ 55)        & {{CATE_OLD}}   & {{SE_OLD}}   & {{CI_LO_OLD}}   & {{CI_HI_OLD}}   & {{N_OLD}}   \\
$p$-value (joint)   & \multicolumn{5}{c}{{{P_AGE_JOINT}}} \\
\addlinespace[0.5em]

\multicolumn{6}{l}{\textit{Panel D: By Education}} \\
\addlinespace[0.3em]
High School or Less & {{CATE_HS}}    & {{SE_HS}}    & {{CI_LO_HS}}    & {{CI_HI_HS}}    & {{N_HS}}    \\
Some College        & {{CATE_SC}}    & {{SE_SC}}    & {{CI_LO_SC}}    & {{CI_HI_SC}}    & {{N_SC}}    \\
College+            & {{CATE_COL}}   & {{SE_COL}}   & {{CI_LO_COL}}   & {{CI_HI_COL}}   & {{N_COL}}   \\
$p$-value (joint)   & \multicolumn{5}{c}{{{P_EDU_JOINT}}} \\
\addlinespace[0.5em]

\multicolumn{6}{l}{\textit{Panel E: By Income Quartile}} \\
\addlinespace[0.3em]
Q1 (Bottom 25\%)    & {{CATE_Q1}}    & {{SE_Q1}}    & {{CI_LO_Q1}}    & {{CI_HI_Q1}}    & {{N_Q1}}    \\
Q2                  & {{CATE_Q2}}    & {{SE_Q2}}    & {{CI_LO_Q2}}    & {{CI_HI_Q2}}    & {{N_Q2}}    \\
Q3                  & {{CATE_Q3}}    & {{SE_Q3}}    & {{CI_LO_Q3}}    & {{CI_HI_Q3}}    & {{N_Q3}}    \\
Q4 (Top 25\%)       & {{CATE_Q4}}    & {{SE_Q4}}    & {{CI_LO_Q4}}    & {{CI_HI_Q4}}    & {{N_Q4}}    \\
$p$-value (Q4 vs Q1)& \multicolumn{5}{c}{{{P_INCOME_DIFF}}} \\

\bottomrule
\end{tabular}

\tablenote{This table reports conditional average treatment effects (CATE) estimated using causal forest.
Panel A shows the average treatment effect (ATE) for the full sample.
Panels B--E show CATEs for pre-specified subgroups.
``Difference'' rows show the difference in CATE between groups.
Joint $p$-values test the null hypothesis that CATEs are equal across all categories.
Standard errors computed using the honest inference procedure of Athey and Imbens (2016).
95\% confidence intervals are constructed using normal approximation.}

\end{threeparttable}
\end{table}


% =============================================================================
% Template 2: Variable Importance and Best Linear Projection
% =============================================================================

\begin{table}[htbp]
\centering
\caption{Treatment Effect Heterogeneity: Variable Importance}
\label{tab:vimp}

\small
\begin{threeparttable}
\begin{tabular}{l
    S[table-format=1.4]     % VIMP
    S[table-format=1.4]     % SE
    c                       % Rank
    S[table-format=-1.3]    % BLP Coef
    S[table-format=1.3]     % BLP SE
}
\toprule
                    & \multicolumn{2}{c}{Variable Importance} & & \multicolumn{2}{c}{Best Linear Projection} \\
\cmidrule(lr){2-3} \cmidrule(lr){5-6}
Variable            & {VIMP} & {SE} & {Rank} & {Coef.} & {SE} \\
\midrule

\multicolumn{6}{l}{\textit{Panel A: Continuous Variables}} \\
\addlinespace[0.3em]
Age                 & 0.1234 & 0.0156 & {1} & 0.008\sym{***} & 0.002 \\
Income (log)        & 0.0987 & 0.0134 & {2} & 0.045\sym{**}  & 0.018 \\
Education (years)   & 0.0756 & 0.0123 & {3} & 0.012\sym{*}   & 0.006 \\
Experience          & 0.0534 & 0.0098 & {4} & -0.003         & 0.004 \\
Hours worked        & 0.0312 & 0.0076 & {6} & 0.002          & 0.001 \\
\addlinespace[0.5em]

\multicolumn{6}{l}{\textit{Panel B: Categorical Variables}} \\
\addlinespace[0.3em]
Female              & 0.0423 & 0.0089 & {5} & -0.089\sym{**} & 0.035 \\
Married             & 0.0234 & 0.0065 & {7} & 0.056\sym{*}   & 0.028 \\
Urban               & 0.0198 & 0.0054 & {8} & 0.034          & 0.025 \\
Has children        & 0.0156 & 0.0043 & {9} & 0.023          & 0.022 \\
\addlinespace[0.5em]

\multicolumn{6}{l}{\textit{Panel C: Interactions (Top 5)}} \\
\addlinespace[0.3em]
Age $\times$ Income & 0.0345 & 0.0078 & {---} &               &       \\
Female $\times$ Education & 0.0289 & 0.0067 & {---} &        &       \\
Age $\times$ Education & 0.0234 & 0.0056 & {---} &           &       \\
Income $\times$ Urban & 0.0198 & 0.0048 & {---} &             &       \\
Female $\times$ Married & 0.0167 & 0.0042 & {---} &           &       \\

\midrule
\multicolumn{6}{l}{\textit{Panel D: Summary Statistics}} \\
\addlinespace[0.3em]
Total VIMP explained & \multicolumn{5}{c}{0.5543} \\
Omnibus test for heterogeneity & \multicolumn{5}{c}{$\chi^2(9) = 45.67$, $p < 0.001$} \\
BLP R-squared       & \multicolumn{5}{c}{0.234} \\

\bottomrule
\end{tabular}

\tablenote{This table reports variable importance for treatment effect heterogeneity.
Columns (1)--(2): Variable importance (VIMP) from causal forest, measuring each variable's contribution to CATE variation.
VIMP is computed as the decrease in out-of-bag prediction accuracy when the variable is permuted.
Column (3): Rank by VIMP magnitude.
Columns (4)--(5): Coefficients from best linear projection (BLP) of estimated CATEs on covariates.
BLP tests whether each variable is linearly associated with treatment effect heterogeneity.
Standard errors for VIMP are computed using bootstrap. \signote\ for BLP coefficients.}

\end{threeparttable}
\end{table}


% =============================================================================
% Template 3: CATE Distribution and Quantiles
% =============================================================================

\begin{table}[htbp]
\centering
\caption{Treatment Effect Distribution}
\label{tab:cate_dist}

\begin{threeparttable}
\begin{tabular}{l*{6}{S[table-format=-1.3]}}
\toprule
                    & {P10}     & {P25}     & {P50}     & {P75}     & {P90}     & {IQR}     \\
\midrule

\multicolumn{7}{l}{\textit{Panel A: CATE Quantiles}} \\
\addlinespace[0.3em]
CATE                & {{P10_CATE}} & {{P25_CATE}} & {{P50_CATE}} & {{P75_CATE}} & {{P90_CATE}} & {{IQR_CATE}} \\
\addlinespace[0.3em]
95\% CI (Lower)     & {{P10_LO}} & {{P25_LO}} & {{P50_LO}} & {{P75_LO}} & {{P90_LO}} & \\
95\% CI (Upper)     & {{P10_HI}} & {{P25_HI}} & {{P50_HI}} & {{P75_HI}} & {{P90_HI}} & \\
\addlinespace[0.5em]

\multicolumn{7}{l}{\textit{Panel B: Share with Positive/Negative Effects}} \\
\addlinespace[0.3em]
Share CATE $>$ 0    & \multicolumn{6}{c}{{{SHARE_POS}} (95\% CI: [{{SHARE_POS_LO}}, {{SHARE_POS_HI}}])} \\
Share CATE $<$ 0    & \multicolumn{6}{c}{{{SHARE_NEG}} (95\% CI: [{{SHARE_NEG_LO}}, {{SHARE_NEG_HI}}])} \\
Share $|$CATE$|$ $>$ 0.1 & \multicolumn{6}{c}{{{SHARE_LARGE}}} \\

\midrule
\multicolumn{7}{l}{\textit{Panel C: Tests for Heterogeneity}} \\
\addlinespace[0.3em]
Variance of CATE    & \multicolumn{6}{c}{{{VAR_CATE}} (SE: {{SE_VAR_CATE}})} \\
$H_0$: No heterogeneity & \multicolumn{6}{c}{$\chi^2 = {{CHI2_HET}}$, $p = {{P_HET}}$} \\
90-10 percentile gap & \multicolumn{6}{c}{{{P90_P10_GAP}} (SE: {{SE_GAP}})} \\

\bottomrule
\end{tabular}

\tablenote{This table reports the distribution of estimated conditional average treatment effects (CATE).
Panel A shows selected quantiles of the CATE distribution across individuals.
Panel B shows the share of individuals with positive vs. negative estimated effects.
Panel C tests whether there is significant heterogeneity in treatment effects.
Variance of CATE is estimated using the decomposition of Athey and Wager (2019).
The null hypothesis of no heterogeneity is tested using the omnibus test from causal forest.}

\end{threeparttable}
\end{table}


% =============================================================================
% Template 4: Policy Targeting and Sorted Effects
% =============================================================================

\begin{table}[htbp]
\centering
\caption{Policy Targeting: Treatment Effect by Predicted CATE Quartile}
\label{tab:targeting}

\begin{threeparttable}
\begin{tabular}{l*{5}{S[table-format=-1.3]}}
\toprule
                    & {Q1 (Lowest)} & {Q2}      & {Q3}      & {Q4 (Highest)} & {Q4$-$Q1} \\
\midrule

\multicolumn{6}{l}{\textit{Panel A: Treatment Effect by Predicted Benefit Quartile}} \\
\addlinespace[0.3em]
Predicted CATE      & -0.045    & 0.123     & 0.245     & 0.412     & 0.457     \\
                    & (0.018)   & (0.025)   & (0.031)   & (0.042)   & (0.045)   \\
\addlinespace[0.3em]
Actual CATE         & -0.023    & 0.098     & 0.198     & 0.356     & 0.379     \\
                    & (0.045)   & (0.052)   & (0.058)   & (0.067)   & (0.081)   \\
\addlinespace[0.5em]

\multicolumn{6}{l}{\textit{Panel B: Targeting Gain Metrics}} \\
\addlinespace[0.3em]
ATE (Treat All)     & \multicolumn{5}{c}{0.156 (SE: 0.028)} \\
LATE (Treat Q4 Only)& \multicolumn{5}{c}{0.356 (SE: 0.067)} \\
Targeting Gain (Q4 vs All) & \multicolumn{5}{c}{0.200 (SE: 0.072)} \\
\addlinespace[0.3em]
AUTOC               & \multicolumn{5}{c}{0.234 (SE: 0.056)} \\
QINI Coefficient    & \multicolumn{5}{c}{0.189 (SE: 0.048)} \\

\midrule
\multicolumn{6}{l}{\textit{Panel C: Calibration}} \\
\addlinespace[0.3em]
Correlation (Pred., Actual) & \multicolumn{5}{c}{0.78} \\
Calibration slope   & \multicolumn{5}{c}{0.86 (SE: 0.12)} \\
Calibration intercept & \multicolumn{5}{c}{0.02 (SE: 0.03)} \\

\midrule
\multicolumn{6}{l}{\textit{Panel D: Sample Characteristics by Quartile}} \\
\addlinespace[0.3em]
Mean Age            & 52.3      & 45.6      & 38.2      & 29.8      &           \\
\% Female           & 0.58      & 0.52      & 0.48      & 0.42      &           \\
Mean Income (log)   & 9.8       & 10.2      & 10.5      & 10.9      &           \\
Mean Education (yrs)& 11.2      & 12.5      & 13.8      & 15.2      &           \\
N                   & 2845      & 2845      & 2845      & 2845      &           \\

\bottomrule
\end{tabular}

\tablenote{This table evaluates the performance of CATE-based targeting.
Panel A: Observations are sorted into quartiles by predicted CATE.
``Predicted CATE'' is the cross-validated forest prediction; ``Actual CATE'' is estimated on held-out folds.
Panel B: AUTOC is the Area Under the Targeting Operator Characteristic curve.
QINI coefficient measures cumulative uplift from targeting.
Panel C: Calibration assesses whether predicted CATEs match actual magnitudes.
Slope = 1 and intercept = 0 indicate perfect calibration.
Panel D: Shows covariate means for each quartile to characterize who benefits most.}

\end{threeparttable}
\end{table}


% =============================================================================
% Template 5: Sorted Group Average Treatment Effects (GATES)
% =============================================================================

\begin{table}[htbp]
\centering
\caption{Sorted Group Average Treatment Effects (GATES)}
\label{tab:gates}

\begin{threeparttable}
\begin{tabular}{l*{5}{S[table-format=-1.3]}c}
\toprule
                    & {Group 1} & {Group 2} & {Group 3} & {Group 4} & {Group 5} & $p$ (het.) \\
                    & {(Bottom 20\%)} & & {(Median)} & & {(Top 20\%)} & \\
\midrule

\multicolumn{7}{l}{\textit{Panel A: GATES Estimates}} \\
\addlinespace[0.3em]
$\gamma_k$ (GATE)   & -0.089    & 0.056     & 0.145     & 0.234     & 0.378     & \\
                    & (0.052)   & (0.048)   & (0.045)   & (0.051)   & (0.062)   & \\
\addlinespace[0.3em]
$\beta_k$ (vs. ATE) & -0.234\sym{***} & -0.089\sym{*} & 0.000 & 0.089\sym{*} & 0.233\sym{***} & $<$0.001 \\
                    & (0.058)   & (0.052)   & {---}     & (0.055)   & (0.068)   & \\
\addlinespace[0.5em]

\multicolumn{7}{l}{\textit{Panel B: Pairwise Comparisons}} \\
\addlinespace[0.3em]
Group 5 $-$ Group 1 & \multicolumn{5}{c}{0.467\sym{***} (SE: 0.081)} & \\
Group 5 $-$ Group 3 & \multicolumn{5}{c}{0.233\sym{***} (SE: 0.076)} & \\
Group 4 $-$ Group 2 & \multicolumn{5}{c}{0.178\sym{**} (SE: 0.070)} & \\
\addlinespace[0.5em]

\multicolumn{7}{l}{\textit{Panel C: ATE in Each Group}} \\
\addlinespace[0.3em]
$E[\tau | G=k]$     & -0.089    & 0.056     & 0.145     & 0.234     & 0.378     & \\
Share of Sample     & {0.200}   & {0.200}   & {0.200}   & {0.200}   & {0.200}   & \\

\midrule
Overall ATE         & \multicolumn{6}{c}{0.145 (SE: 0.028)} \\
Observations        & \multicolumn{6}{c}{14,280} \\

\bottomrule
\end{tabular}

\tablenote{This table reports Sorted Group Average Treatment Effects (GATES) following Chernozhukov et al. (2020).
Panel A: $\gamma_k$ is the average treatment effect for observations in group $k$ (sorted by predicted CATE).
$\beta_k$ is the deviation of $\gamma_k$ from the overall ATE; $\beta_3 = 0$ by construction (median group).
Panel B: Tests for differences between selected groups.
The $p$-value for heterogeneity tests the null that all $\beta_k = 0$ (no heterogeneity).
Standard errors are computed using the method of Chernozhukov et al. (2020) with cross-fitting.}

\end{threeparttable}
\end{table}

\end{document}

% =============================================================================
% USAGE NOTES:
% =============================================================================
%
% 1. CATE Estimation Methods:
%    - Causal Forest (Athey & Wager, 2019): grf package in R
%    - DR-Learner (Kennedy, 2020): econml in Python
%    - T-Learner, S-Learner, X-Learner: causalml package
%    - Double Machine Learning: doubleml package
%
% 2. Variable Importance Interpretation:
%    - VIMP measures contribution to CATE variation, not prediction
%    - Higher VIMP = more treatment effect heterogeneity along that dimension
%    - BLP provides linear approximation of heterogeneity direction
%
% 3. Honest Inference:
%    - Split sample: training vs. estimation trees
%    - Cross-fitting for efficiency
%    - Bootstrap or analytical SEs available
%
% 4. Policy Targeting Metrics:
%    - AUTOC: Area Under Targeting Operating Characteristic
%    - QINI: Cumulative uplift from targeting
%    - Targeting gain = LATE(Top K%) - ATE
%
% 5. GATES (Chernozhukov et al., 2020):
%    - Groups formed by sorting predicted CATEs
%    - Tests for heterogeneity using parametric bootstrap
%    - Robust to model misspecification via cross-fitting
%
% 6. Software Commands:
%    - R: grf::causal_forest(), grf::average_treatment_effect()
%    - Python: econml.dml.CausalForestDML
%    - Python: causalml.inference.meta.XGBTRegressor
%
% 7. Reporting Best Practices:
%    - Always report sample sizes for each subgroup
%    - Show both predicted and realized CATEs for targeting tables
%    - Include calibration metrics when claiming targeting value
%    - Test for pre-specified (theory-driven) heterogeneity
%    - Report data-driven (ML) heterogeneity separately
% =============================================================================
