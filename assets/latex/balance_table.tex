% =============================================================================
% balance_table.tex - PSM Balance Table Template
% =============================================================================
% Style: AER/NBER publication standards for propensity score matching
% Features:
%   - Before/after matching comparison
%   - Standardized mean differences
%   - Variance ratios
%   - Balance diagnostics (normalized differences, t-tests)
%   - Multiple matching methods comparison
%
% Placeholders (replace with your data):
%   - {{TITLE}}          : Table title
%   - {{LABEL}}          : LaTeX label for cross-referencing
%   - {{VAR_*}}          : Variable names
%   - {{*_BEFORE}}       : Pre-matching statistics
%   - {{*_AFTER}}        : Post-matching statistics
%   - {{NORMDIFF_*}}     : Normalized differences
%   - {{PVAL_*}}         : Balance test p-values
% =============================================================================

\documentclass[12pt]{article}
% =============================================================================
% common_preamble.tex - Shared LaTeX Preamble for Causal Inference Tables/Figures
% =============================================================================
% Usage: % =============================================================================
% common_preamble.tex - Shared LaTeX Preamble for Causal Inference Tables/Figures
% =============================================================================
% Usage: % =============================================================================
% common_preamble.tex - Shared LaTeX Preamble for Causal Inference Tables/Figures
% =============================================================================
% Usage: \input{common_preamble.tex} at the beginning of your document
% Compatible with: pdflatex, xelatex, lualatex
% Style: AER/NBER publication standards
% =============================================================================

% -----------------------------------------------------------------------------
% Required Packages
% -----------------------------------------------------------------------------

% Tables
\usepackage{booktabs}           % Professional table rules (\toprule, \midrule, \bottomrule)
\usepackage{threeparttable}     % Tables with notes
\usepackage{multirow}           % Multi-row cells
\usepackage{array}              % Enhanced column specifications
\usepackage{tabularx}           % Auto-width columns
\usepackage{longtable}          % Multi-page tables

% Number Formatting
\usepackage{siunitx}            % Consistent number formatting and alignment
\sisetup{
    input-symbols = (),         % Allow parentheses in input
    input-open-uncertainty  = , % No uncertainty notation
    input-close-uncertainty = ,
    table-align-text-pre    = false,
    table-align-text-post   = false,
    group-separator         = {,},      % Thousands separator
    group-minimum-digits    = 4,        % Apply grouping for 4+ digits
    round-mode              = places,   % Round to decimal places
    round-precision         = 3,        % Default 3 decimal places
    detect-weight           = true,     % Detect bold
    detect-family           = true,     % Detect font family
}

% Graphics and Plots
\usepackage{graphicx}           % Include graphics
\usepackage{pgfplots}           % Publication-quality plots
\pgfplotsset{compat=1.18}       % Use latest pgfplots features
\usepgfplotslibrary{fillbetween}% For confidence interval shading
\usetikzlibrary{patterns}       % Pattern fills
\usetikzlibrary{arrows.meta}    % Arrow styles

% Colors
\usepackage{xcolor}             % Color support

% Captions
\usepackage{caption}            % Caption customization
\captionsetup{
    font=small,
    labelfont=bf,
    skip=10pt
}

% -----------------------------------------------------------------------------
% Color Definitions (Publication-Friendly Palette)
% -----------------------------------------------------------------------------

% Primary colors for plots
\definecolor{coefblue}{RGB}{31, 119, 180}       % Main coefficient color
\definecolor{ciblue}{RGB}{174, 199, 232}        % Confidence interval fill
\definecolor{treatred}{RGB}{214, 39, 40}        % Treatment/significant
\definecolor{controlgreen}{RGB}{44, 160, 44}    % Control group
\definecolor{refgray}{RGB}{127, 127, 127}       % Reference lines
\definecolor{shadegray}{RGB}{220, 220, 220}     % Shaded regions

% Secondary colors
\definecolor{orange}{RGB}{255, 127, 14}
\definecolor{purple}{RGB}{148, 103, 189}
\definecolor{brown}{RGB}{140, 86, 75}
\definecolor{pink}{RGB}{227, 119, 194}

% -----------------------------------------------------------------------------
% Significance Stars Commands
% -----------------------------------------------------------------------------

% Standard significance stars (AER style)
% Usage: \sym{*}, \sym{**}, \sym{***}
\newcommand{\sym}[1]{\ensuremath{^{#1}}}

% Significance level commands with standard thresholds
% Usage: \stars{0.001} produces ***
\newcommand{\stars}[1]{%
    \ifnum\pdfstrcmp{#1}{0.01}<0 \sym{***}%
    \else\ifnum\pdfstrcmp{#1}{0.05}<0 \sym{**}%
    \else\ifnum\pdfstrcmp{#1}{0.1}<0 \sym{*}%
    \fi\fi\fi%
}

% Pre-defined star commands
\newcommand{\onestar}{\sym{*}}
\newcommand{\twostar}{\sym{**}}
\newcommand{\threestar}{\sym{***}}

% -----------------------------------------------------------------------------
% Table Formatting Commands
% -----------------------------------------------------------------------------

% Standard error in parentheses
% Usage: \se{0.123}
\newcommand{\se}[1]{(#1)}

% Coefficient with standard error
% Usage: \coefse{1.234}{0.123}
\newcommand{\coefse}[2]{#1 \\ \se{#2}}

% Coefficient with SE and stars
% Usage: \coefstars{1.234}{0.123}{**}
\newcommand{\coefstars}[3]{#1\sym{#3} \\ \se{#2}}

% Yes/No indicators for fixed effects
\newcommand{\yes}{Yes}
\newcommand{\no}{No}
\newcommand{\ymark}{\checkmark}

% Panel headers
\newcommand{\panelheader}[1]{\multicolumn{\numcols}{l}{\textit{Panel #1}}}

% -----------------------------------------------------------------------------
% Number Formatting Commands
% -----------------------------------------------------------------------------

% Format large numbers with comma separators
% Usage: \numfmt{12345} produces 12,345
\newcommand{\numfmt}[1]{\num[group-separator={,}]{#1}}

% Format percentages
% Usage: \pct{45.6} produces 45.6%
\newcommand{\pct}[1]{#1\%}

% Format p-values
% Usage: \pval{0.001} produces <0.001 for very small values
\newcommand{\pval}[1]{%
    \ifdim #1 pt < 0.001 pt
        $<$0.001%
    \else
        #1%
    \fi
}

% -----------------------------------------------------------------------------
% Note Formatting for Tables
% -----------------------------------------------------------------------------

% Standard table notes style
\newcommand{\tablenote}[1]{%
    \begin{tablenotes}[flushleft]
        \small
        \item \textit{Notes:} #1
    \end{tablenotes}
}

% Significance note (standard)
\newcommand{\signote}{%
    \sym{*} $p<0.10$, \sym{**} $p<0.05$, \sym{***} $p<0.01$%
}

% Full standard note
\newcommand{\stdnote}[1]{%
    \tablenote{#1 Standard errors in parentheses. \signote.}%
}

% -----------------------------------------------------------------------------
% PGFPlots Style Definitions
% -----------------------------------------------------------------------------

\pgfplotsset{
    % Style for coefficient plots
    coefplot/.style={
        width=0.9\textwidth,
        height=0.6\textwidth,
        xbar,
        bar width=3pt,
        xmin=-2, xmax=2,
        axis lines=left,
        ytick=data,
        enlarge y limits=0.15,
        xlabel={Coefficient Estimate},
        every axis x label/.style={
            at={(ticklabel* cs:1.02)},
            anchor=west,
        },
        x axis line style={-},
        y axis line style={opacity=0},
        tickwidth=0pt,
        clip=false,
    },
    % Style for event study plots
    eventstudy/.style={
        width=0.9\textwidth,
        height=0.5\textwidth,
        axis lines=middle,
        xlabel={Time Relative to Treatment},
        ylabel={Coefficient},
        xmin=-6, xmax=6,
        xtick={-5,-4,-3,-2,-1,0,1,2,3,4,5},
        extra x ticks={0},
        extra x tick style={grid=major, grid style={dashed, red!50}},
        legend pos=north west,
        legend style={draw=none, fill=none},
    },
    % Common style elements
    every axis/.append style={
        font=\small,
        tick label style={font=\small},
        label style={font=\small},
        legend style={font=\small},
    },
}

% -----------------------------------------------------------------------------
% End of Preamble
% -----------------------------------------------------------------------------
 at the beginning of your document
% Compatible with: pdflatex, xelatex, lualatex
% Style: AER/NBER publication standards
% =============================================================================

% -----------------------------------------------------------------------------
% Required Packages
% -----------------------------------------------------------------------------

% Tables
\usepackage{booktabs}           % Professional table rules (\toprule, \midrule, \bottomrule)
\usepackage{threeparttable}     % Tables with notes
\usepackage{multirow}           % Multi-row cells
\usepackage{array}              % Enhanced column specifications
\usepackage{tabularx}           % Auto-width columns
\usepackage{longtable}          % Multi-page tables

% Number Formatting
\usepackage{siunitx}            % Consistent number formatting and alignment
\sisetup{
    input-symbols = (),         % Allow parentheses in input
    input-open-uncertainty  = , % No uncertainty notation
    input-close-uncertainty = ,
    table-align-text-pre    = false,
    table-align-text-post   = false,
    group-separator         = {,},      % Thousands separator
    group-minimum-digits    = 4,        % Apply grouping for 4+ digits
    round-mode              = places,   % Round to decimal places
    round-precision         = 3,        % Default 3 decimal places
    detect-weight           = true,     % Detect bold
    detect-family           = true,     % Detect font family
}

% Graphics and Plots
\usepackage{graphicx}           % Include graphics
\usepackage{pgfplots}           % Publication-quality plots
\pgfplotsset{compat=1.18}       % Use latest pgfplots features
\usepgfplotslibrary{fillbetween}% For confidence interval shading
\usetikzlibrary{patterns}       % Pattern fills
\usetikzlibrary{arrows.meta}    % Arrow styles

% Colors
\usepackage{xcolor}             % Color support

% Captions
\usepackage{caption}            % Caption customization
\captionsetup{
    font=small,
    labelfont=bf,
    skip=10pt
}

% -----------------------------------------------------------------------------
% Color Definitions (Publication-Friendly Palette)
% -----------------------------------------------------------------------------

% Primary colors for plots
\definecolor{coefblue}{RGB}{31, 119, 180}       % Main coefficient color
\definecolor{ciblue}{RGB}{174, 199, 232}        % Confidence interval fill
\definecolor{treatred}{RGB}{214, 39, 40}        % Treatment/significant
\definecolor{controlgreen}{RGB}{44, 160, 44}    % Control group
\definecolor{refgray}{RGB}{127, 127, 127}       % Reference lines
\definecolor{shadegray}{RGB}{220, 220, 220}     % Shaded regions

% Secondary colors
\definecolor{orange}{RGB}{255, 127, 14}
\definecolor{purple}{RGB}{148, 103, 189}
\definecolor{brown}{RGB}{140, 86, 75}
\definecolor{pink}{RGB}{227, 119, 194}

% -----------------------------------------------------------------------------
% Significance Stars Commands
% -----------------------------------------------------------------------------

% Standard significance stars (AER style)
% Usage: \sym{*}, \sym{**}, \sym{***}
\newcommand{\sym}[1]{\ensuremath{^{#1}}}

% Significance level commands with standard thresholds
% Usage: \stars{0.001} produces ***
\newcommand{\stars}[1]{%
    \ifnum\pdfstrcmp{#1}{0.01}<0 \sym{***}%
    \else\ifnum\pdfstrcmp{#1}{0.05}<0 \sym{**}%
    \else\ifnum\pdfstrcmp{#1}{0.1}<0 \sym{*}%
    \fi\fi\fi%
}

% Pre-defined star commands
\newcommand{\onestar}{\sym{*}}
\newcommand{\twostar}{\sym{**}}
\newcommand{\threestar}{\sym{***}}

% -----------------------------------------------------------------------------
% Table Formatting Commands
% -----------------------------------------------------------------------------

% Standard error in parentheses
% Usage: \se{0.123}
\newcommand{\se}[1]{(#1)}

% Coefficient with standard error
% Usage: \coefse{1.234}{0.123}
\newcommand{\coefse}[2]{#1 \\ \se{#2}}

% Coefficient with SE and stars
% Usage: \coefstars{1.234}{0.123}{**}
\newcommand{\coefstars}[3]{#1\sym{#3} \\ \se{#2}}

% Yes/No indicators for fixed effects
\newcommand{\yes}{Yes}
\newcommand{\no}{No}
\newcommand{\ymark}{\checkmark}

% Panel headers
\newcommand{\panelheader}[1]{\multicolumn{\numcols}{l}{\textit{Panel #1}}}

% -----------------------------------------------------------------------------
% Number Formatting Commands
% -----------------------------------------------------------------------------

% Format large numbers with comma separators
% Usage: \numfmt{12345} produces 12,345
\newcommand{\numfmt}[1]{\num[group-separator={,}]{#1}}

% Format percentages
% Usage: \pct{45.6} produces 45.6%
\newcommand{\pct}[1]{#1\%}

% Format p-values
% Usage: \pval{0.001} produces <0.001 for very small values
\newcommand{\pval}[1]{%
    \ifdim #1 pt < 0.001 pt
        $<$0.001%
    \else
        #1%
    \fi
}

% -----------------------------------------------------------------------------
% Note Formatting for Tables
% -----------------------------------------------------------------------------

% Standard table notes style
\newcommand{\tablenote}[1]{%
    \begin{tablenotes}[flushleft]
        \small
        \item \textit{Notes:} #1
    \end{tablenotes}
}

% Significance note (standard)
\newcommand{\signote}{%
    \sym{*} $p<0.10$, \sym{**} $p<0.05$, \sym{***} $p<0.01$%
}

% Full standard note
\newcommand{\stdnote}[1]{%
    \tablenote{#1 Standard errors in parentheses. \signote.}%
}

% -----------------------------------------------------------------------------
% PGFPlots Style Definitions
% -----------------------------------------------------------------------------

\pgfplotsset{
    % Style for coefficient plots
    coefplot/.style={
        width=0.9\textwidth,
        height=0.6\textwidth,
        xbar,
        bar width=3pt,
        xmin=-2, xmax=2,
        axis lines=left,
        ytick=data,
        enlarge y limits=0.15,
        xlabel={Coefficient Estimate},
        every axis x label/.style={
            at={(ticklabel* cs:1.02)},
            anchor=west,
        },
        x axis line style={-},
        y axis line style={opacity=0},
        tickwidth=0pt,
        clip=false,
    },
    % Style for event study plots
    eventstudy/.style={
        width=0.9\textwidth,
        height=0.5\textwidth,
        axis lines=middle,
        xlabel={Time Relative to Treatment},
        ylabel={Coefficient},
        xmin=-6, xmax=6,
        xtick={-5,-4,-3,-2,-1,0,1,2,3,4,5},
        extra x ticks={0},
        extra x tick style={grid=major, grid style={dashed, red!50}},
        legend pos=north west,
        legend style={draw=none, fill=none},
    },
    % Common style elements
    every axis/.append style={
        font=\small,
        tick label style={font=\small},
        label style={font=\small},
        legend style={font=\small},
    },
}

% -----------------------------------------------------------------------------
% End of Preamble
% -----------------------------------------------------------------------------
 at the beginning of your document
% Compatible with: pdflatex, xelatex, lualatex
% Style: AER/NBER publication standards
% =============================================================================

% -----------------------------------------------------------------------------
% Required Packages
% -----------------------------------------------------------------------------

% Tables
\usepackage{booktabs}           % Professional table rules (\toprule, \midrule, \bottomrule)
\usepackage{threeparttable}     % Tables with notes
\usepackage{multirow}           % Multi-row cells
\usepackage{array}              % Enhanced column specifications
\usepackage{tabularx}           % Auto-width columns
\usepackage{longtable}          % Multi-page tables

% Number Formatting
\usepackage{siunitx}            % Consistent number formatting and alignment
\sisetup{
    input-symbols = (),         % Allow parentheses in input
    input-open-uncertainty  = , % No uncertainty notation
    input-close-uncertainty = ,
    table-align-text-pre    = false,
    table-align-text-post   = false,
    group-separator         = {,},      % Thousands separator
    group-minimum-digits    = 4,        % Apply grouping for 4+ digits
    round-mode              = places,   % Round to decimal places
    round-precision         = 3,        % Default 3 decimal places
    detect-weight           = true,     % Detect bold
    detect-family           = true,     % Detect font family
}

% Graphics and Plots
\usepackage{graphicx}           % Include graphics
\usepackage{pgfplots}           % Publication-quality plots
\pgfplotsset{compat=1.18}       % Use latest pgfplots features
\usepgfplotslibrary{fillbetween}% For confidence interval shading
\usetikzlibrary{patterns}       % Pattern fills
\usetikzlibrary{arrows.meta}    % Arrow styles

% Colors
\usepackage{xcolor}             % Color support

% Captions
\usepackage{caption}            % Caption customization
\captionsetup{
    font=small,
    labelfont=bf,
    skip=10pt
}

% -----------------------------------------------------------------------------
% Color Definitions (Publication-Friendly Palette)
% -----------------------------------------------------------------------------

% Primary colors for plots
\definecolor{coefblue}{RGB}{31, 119, 180}       % Main coefficient color
\definecolor{ciblue}{RGB}{174, 199, 232}        % Confidence interval fill
\definecolor{treatred}{RGB}{214, 39, 40}        % Treatment/significant
\definecolor{controlgreen}{RGB}{44, 160, 44}    % Control group
\definecolor{refgray}{RGB}{127, 127, 127}       % Reference lines
\definecolor{shadegray}{RGB}{220, 220, 220}     % Shaded regions

% Secondary colors
\definecolor{orange}{RGB}{255, 127, 14}
\definecolor{purple}{RGB}{148, 103, 189}
\definecolor{brown}{RGB}{140, 86, 75}
\definecolor{pink}{RGB}{227, 119, 194}

% -----------------------------------------------------------------------------
% Significance Stars Commands
% -----------------------------------------------------------------------------

% Standard significance stars (AER style)
% Usage: \sym{*}, \sym{**}, \sym{***}
\newcommand{\sym}[1]{\ensuremath{^{#1}}}

% Significance level commands with standard thresholds
% Usage: \stars{0.001} produces ***
\newcommand{\stars}[1]{%
    \ifnum\pdfstrcmp{#1}{0.01}<0 \sym{***}%
    \else\ifnum\pdfstrcmp{#1}{0.05}<0 \sym{**}%
    \else\ifnum\pdfstrcmp{#1}{0.1}<0 \sym{*}%
    \fi\fi\fi%
}

% Pre-defined star commands
\newcommand{\onestar}{\sym{*}}
\newcommand{\twostar}{\sym{**}}
\newcommand{\threestar}{\sym{***}}

% -----------------------------------------------------------------------------
% Table Formatting Commands
% -----------------------------------------------------------------------------

% Standard error in parentheses
% Usage: \se{0.123}
\newcommand{\se}[1]{(#1)}

% Coefficient with standard error
% Usage: \coefse{1.234}{0.123}
\newcommand{\coefse}[2]{#1 \\ \se{#2}}

% Coefficient with SE and stars
% Usage: \coefstars{1.234}{0.123}{**}
\newcommand{\coefstars}[3]{#1\sym{#3} \\ \se{#2}}

% Yes/No indicators for fixed effects
\newcommand{\yes}{Yes}
\newcommand{\no}{No}
\newcommand{\ymark}{\checkmark}

% Panel headers
\newcommand{\panelheader}[1]{\multicolumn{\numcols}{l}{\textit{Panel #1}}}

% -----------------------------------------------------------------------------
% Number Formatting Commands
% -----------------------------------------------------------------------------

% Format large numbers with comma separators
% Usage: \numfmt{12345} produces 12,345
\newcommand{\numfmt}[1]{\num[group-separator={,}]{#1}}

% Format percentages
% Usage: \pct{45.6} produces 45.6%
\newcommand{\pct}[1]{#1\%}

% Format p-values
% Usage: \pval{0.001} produces <0.001 for very small values
\newcommand{\pval}[1]{%
    \ifdim #1 pt < 0.001 pt
        $<$0.001%
    \else
        #1%
    \fi
}

% -----------------------------------------------------------------------------
% Note Formatting for Tables
% -----------------------------------------------------------------------------

% Standard table notes style
\newcommand{\tablenote}[1]{%
    \begin{tablenotes}[flushleft]
        \small
        \item \textit{Notes:} #1
    \end{tablenotes}
}

% Significance note (standard)
\newcommand{\signote}{%
    \sym{*} $p<0.10$, \sym{**} $p<0.05$, \sym{***} $p<0.01$%
}

% Full standard note
\newcommand{\stdnote}[1]{%
    \tablenote{#1 Standard errors in parentheses. \signote.}%
}

% -----------------------------------------------------------------------------
% PGFPlots Style Definitions
% -----------------------------------------------------------------------------

\pgfplotsset{
    % Style for coefficient plots
    coefplot/.style={
        width=0.9\textwidth,
        height=0.6\textwidth,
        xbar,
        bar width=3pt,
        xmin=-2, xmax=2,
        axis lines=left,
        ytick=data,
        enlarge y limits=0.15,
        xlabel={Coefficient Estimate},
        every axis x label/.style={
            at={(ticklabel* cs:1.02)},
            anchor=west,
        },
        x axis line style={-},
        y axis line style={opacity=0},
        tickwidth=0pt,
        clip=false,
    },
    % Style for event study plots
    eventstudy/.style={
        width=0.9\textwidth,
        height=0.5\textwidth,
        axis lines=middle,
        xlabel={Time Relative to Treatment},
        ylabel={Coefficient},
        xmin=-6, xmax=6,
        xtick={-5,-4,-3,-2,-1,0,1,2,3,4,5},
        extra x ticks={0},
        extra x tick style={grid=major, grid style={dashed, red!50}},
        legend pos=north west,
        legend style={draw=none, fill=none},
    },
    % Common style elements
    every axis/.append style={
        font=\small,
        tick label style={font=\small},
        label style={font=\small},
        legend style={font=\small},
    },
}

% -----------------------------------------------------------------------------
% End of Preamble
% -----------------------------------------------------------------------------


\begin{document}

% =============================================================================
% Template 1: Comprehensive Balance Table (Before/After Matching)
% =============================================================================

\begin{table}[htbp]
\centering
\caption{{{TITLE}}: Covariate Balance Before and After Matching}
\label{tab:{{LABEL}}_balance}

\small
\begin{threeparttable}
\begin{tabular}{l
    S[table-format=2.3]     % Treated Mean (Before)
    S[table-format=2.3]     % Control Mean (Before)
    S[table-format=-1.3]    % Std. Diff (Before)
    S[table-format=2.3]     % Treated Mean (After)
    S[table-format=2.3]     % Control Mean (After)
    S[table-format=-1.3]    % Std. Diff (After)
    S[table-format=1.2]     % Variance Ratio (After)
}
\toprule
                    & \multicolumn{3}{c}{Before Matching} & \multicolumn{4}{c}{After Matching} \\
\cmidrule(lr){2-4} \cmidrule(lr){5-8}
Variable            & {Treated} & {Control} & {Std. Diff} & {Treated} & {Control} & {Std. Diff} & {Var. Ratio} \\
\midrule

% Demographic characteristics
\multicolumn{8}{l}{\textit{Panel A: Demographics}} \\
\addlinespace[0.3em]
Age                 & {{T_MEAN_AGE_BEFORE}}  & {{C_MEAN_AGE_BEFORE}}  & {{STDDIFF_AGE_BEFORE}}  & {{T_MEAN_AGE_AFTER}}  & {{C_MEAN_AGE_AFTER}}  & {{STDDIFF_AGE_AFTER}}  & {{VARRATIO_AGE}}  \\
Female              & {{T_MEAN_FEM_BEFORE}}  & {{C_MEAN_FEM_BEFORE}}  & {{STDDIFF_FEM_BEFORE}}  & {{T_MEAN_FEM_AFTER}}  & {{C_MEAN_FEM_AFTER}}  & {{STDDIFF_FEM_AFTER}}  & {{VARRATIO_FEM}}  \\
Education (years)   & {{T_MEAN_EDU_BEFORE}}  & {{C_MEAN_EDU_BEFORE}}  & {{STDDIFF_EDU_BEFORE}}  & {{T_MEAN_EDU_AFTER}}  & {{C_MEAN_EDU_AFTER}}  & {{STDDIFF_EDU_AFTER}}  & {{VARRATIO_EDU}}  \\
Married             & {{T_MEAN_MAR_BEFORE}}  & {{C_MEAN_MAR_BEFORE}}  & {{STDDIFF_MAR_BEFORE}}  & {{T_MEAN_MAR_AFTER}}  & {{C_MEAN_MAR_AFTER}}  & {{STDDIFF_MAR_AFTER}}  & {{VARRATIO_MAR}}  \\
\addlinespace[0.5em]

% Economic characteristics
\multicolumn{8}{l}{\textit{Panel B: Economic Characteristics}} \\
\addlinespace[0.3em]
Income (log)        & {{T_MEAN_INC_BEFORE}}  & {{C_MEAN_INC_BEFORE}}  & {{STDDIFF_INC_BEFORE}}  & {{T_MEAN_INC_AFTER}}  & {{C_MEAN_INC_AFTER}}  & {{STDDIFF_INC_AFTER}}  & {{VARRATIO_INC}}  \\
Employed            & {{T_MEAN_EMP_BEFORE}}  & {{C_MEAN_EMP_BEFORE}}  & {{STDDIFF_EMP_BEFORE}}  & {{T_MEAN_EMP_AFTER}}  & {{C_MEAN_EMP_AFTER}}  & {{STDDIFF_EMP_AFTER}}  & {{VARRATIO_EMP}}  \\
Wealth index        & {{T_MEAN_WLT_BEFORE}}  & {{C_MEAN_WLT_BEFORE}}  & {{STDDIFF_WLT_BEFORE}}  & {{T_MEAN_WLT_AFTER}}  & {{C_MEAN_WLT_AFTER}}  & {{STDDIFF_WLT_AFTER}}  & {{VARRATIO_WLT}}  \\
\addlinespace[0.5em]

% Pre-treatment outcomes
\multicolumn{8}{l}{\textit{Panel C: Pre-Treatment Outcomes}} \\
\addlinespace[0.3em]
Outcome$_{t-1}$     & {{T_MEAN_LAG_BEFORE}}  & {{C_MEAN_LAG_BEFORE}}  & {{STDDIFF_LAG_BEFORE}}  & {{T_MEAN_LAG_AFTER}}  & {{C_MEAN_LAG_AFTER}}  & {{STDDIFF_LAG_AFTER}}  & {{VARRATIO_LAG}}  \\
Outcome$_{t-2}$     & {{T_MEAN_LAG2_BEFORE}} & {{C_MEAN_LAG2_BEFORE}} & {{STDDIFF_LAG2_BEFORE}} & {{T_MEAN_LAG2_AFTER}} & {{C_MEAN_LAG2_AFTER}} & {{STDDIFF_LAG2_AFTER}} & {{VARRATIO_LAG2}} \\

\midrule
\multicolumn{8}{l}{\textit{Sample Size}} \\
\addlinespace[0.3em]
Treated             & \multicolumn{3}{c}{{{N_TREATED_BEFORE}}} & \multicolumn{4}{c}{{{N_TREATED_AFTER}}} \\
Control             & \multicolumn{3}{c}{{{N_CONTROL_BEFORE}}} & \multicolumn{4}{c}{{{N_CONTROL_AFTER}}} \\

\midrule
\multicolumn{8}{l}{\textit{Overall Balance}} \\
\addlinespace[0.3em]
Mean $|$Std. Diff$|$  & \multicolumn{3}{c}{{{MEAN_STDDIFF_BEFORE}}} & \multicolumn{4}{c}{{{MEAN_STDDIFF_AFTER}}} \\
Median $|$Std. Diff$|$  & \multicolumn{3}{c}{{{MED_STDDIFF_BEFORE}}} & \multicolumn{4}{c}{{{MED_STDDIFF_AFTER}}} \\
Rubin's B           & \multicolumn{3}{c}{{{RUBIN_B_BEFORE}}} & \multicolumn{4}{c}{{{RUBIN_B_AFTER}}} \\
Rubin's R           & \multicolumn{3}{c}{{{RUBIN_R_BEFORE}}} & \multicolumn{4}{c}{{{RUBIN_R_AFTER}}} \\

\bottomrule
\end{tabular}

\tablenote{This table presents covariate balance before and after propensity score matching.
Standardized differences are calculated as $(\bar{X}_T - \bar{X}_C) / \sqrt{(s_T^2 + s_C^2)/2}$.
Variance ratios are the ratio of treated to control group variances after matching.
Balance thresholds: $|\text{Std. Diff}| < 0.1$ (Rosenbaum \& Rubin, 1985); Variance Ratio $\in [0.5, 2]$.
Rubin's B measures the absolute standardized difference of the linear index of the propensity score.
Rubin's R is the ratio of treated to control variances of the propensity score index; $B < 25$ and $0.5 < R < 2$ indicate sufficient balance.}

\end{threeparttable}
\end{table}


% =============================================================================
% Template 2: Matching Method Comparison Table
% =============================================================================

\begin{table}[htbp]
\centering
\caption{Balance Comparison Across Matching Methods}
\label{tab:balance_methods}

\small
\begin{threeparttable}
\begin{tabular}{l
    S[table-format=-1.3]    % Unmatched
    S[table-format=-1.3]    % Nearest Neighbor
    S[table-format=-1.3]    % Caliper
    S[table-format=-1.3]    % Kernel
    S[table-format=-1.3]    % IPW
}
\toprule
                    & {Unmatched} & {Nearest} & {Caliper} & {Kernel} & {IPW} \\
                    &             & {Neighbor} & {(0.05)}  &          &       \\
\midrule

\multicolumn{6}{l}{\textit{Standardized Mean Differences}} \\
\addlinespace[0.3em]
Age                 & 0.352  & 0.045  & 0.032  & 0.028  & 0.041  \\
Female              & 0.128  & 0.018  & 0.015  & 0.012  & 0.022  \\
Education           & 0.456  & 0.062  & 0.048  & 0.035  & 0.058  \\
Income (log)        & 0.523  & 0.078  & 0.056  & 0.042  & 0.071  \\
Employed            & 0.234  & 0.028  & 0.021  & 0.018  & 0.032  \\
\addlinespace[0.5em]

\multicolumn{6}{l}{\textit{Summary Statistics}} \\
\addlinespace[0.3em]
Mean $|$Std. Diff$|$  & {0.339}  & {0.046}  & {0.034}  & {0.027}  & {0.045}  \\
Max $|$Std. Diff$|$   & {0.523}  & {0.078}  & {0.056}  & {0.042}  & {0.071}  \\
Covariates $> 0.1$    & {5}      & {0}      & {0}      & {0}      & {0}      \\
\addlinespace[0.3em]

N (Treated)         & {1245}   & {1245}   & {1198}   & {1245}   & {1245}   \\
N (Control)         & {3567}   & {1245}   & {1198}   & {3567}   & {3567}   \\
\% Treated Matched  & {100.0}  & {100.0}  & {96.2}   & {100.0}  & {100.0}  \\

\bottomrule
\end{tabular}

\tablenote{This table compares covariate balance across different matching/weighting methods.
Nearest Neighbor: 1:1 matching without replacement.
Caliper: 1:1 matching with 0.05 caliper on propensity score.
Kernel: Epanechnikov kernel with bandwidth 0.06.
IPW: Inverse probability weighting with normalized weights.
All methods use the same propensity score model: logit with all covariates.}

\end{threeparttable}
\end{table}


% =============================================================================
% Template 3: Love Plot (Coefficient Plot Style for Balance)
% =============================================================================

\begin{figure}[htbp]
\centering
\caption{Covariate Balance: Love Plot}
\label{fig:love_plot}

\begin{tikzpicture}
\begin{axis}[
    width=0.85\textwidth,
    height=0.55\textwidth,
    xbar,
    bar width=0pt,  % No bars, just points
    xmin=-0.6, xmax=0.6,
    xlabel={Standardized Mean Difference},
    ytick={1,2,3,4,5,6,7,8},
    yticklabels={
        {Outcome$_{t-2}$},
        {Outcome$_{t-1}$},
        {Wealth index},
        {Employed},
        {Income (log)},
        {Education},
        {Female},
        {Age}
    },
    ymin=0.5, ymax=8.5,
    enlarge y limits=0.08,
    axis lines=left,
    x axis line style={-},
    y axis line style={opacity=0},
    tickwidth=0pt,
    xmajorgrids=true,
    grid style={dashed, gray!30},
    % Threshold bands
    extra x ticks={-0.1, 0.1},
    extra x tick style={
        grid=major,
        grid style={solid, controlgreen!50, line width=1pt},
        tick label style={opacity=0}
    },
    % Reference line at zero
    extra x ticks={0},
    extra x tick style={
        grid=major,
        grid style={solid, black, line width=0.5pt},
        tick label style={opacity=0}
    },
    legend style={
        at={(0.98,0.02)},
        anchor=south east,
        draw=none,
        fill=white,
        fill opacity=0.9,
    },
    clip=false,
]

% Threshold region shading
\fill[controlgreen, opacity=0.1] (axis cs:-0.1,0.5) rectangle (axis cs:0.1,8.5);

% Before matching (squares)
\addplot[
    only marks,
    mark=square*,
    mark size=4pt,
    treatred,
] coordinates {
    (0.35, 8)   % Age
    (0.13, 7)   % Female
    (0.46, 6)   % Education
    (0.52, 5)   % Income
    (0.23, 4)   % Employed
    (0.38, 3)   % Wealth
    (0.28, 2)   % Outcome t-1
    (0.31, 1)   % Outcome t-2
};
\addlegendentry{Before Matching}

% After matching (circles)
\addplot[
    only marks,
    mark=*,
    mark size=4pt,
    coefblue,
] coordinates {
    (0.04, 8)   % Age
    (0.02, 7)   % Female
    (0.06, 6)   % Education
    (0.08, 5)   % Income
    (0.03, 4)   % Employed
    (0.04, 3)   % Wealth
    (0.02, 2)   % Outcome t-1
    (0.03, 1)   % Outcome t-2
};
\addlegendentry{After Matching}

% Connecting lines
\foreach \y/\xbefore/\xafter in {8/0.35/0.04, 7/0.13/0.02, 6/0.46/0.06, 5/0.52/0.08, 4/0.23/0.03, 3/0.38/0.04, 2/0.28/0.02, 1/0.31/0.03} {
    \draw[gray, opacity=0.5, line width=0.5pt] (axis cs:\xbefore,\y) -- (axis cs:\xafter,\y);
}

% Threshold annotations
\node[anchor=south, font=\scriptsize, controlgreen!70!black] at (axis cs:0.1,8.7) {0.1};
\node[anchor=south, font=\scriptsize, controlgreen!70!black] at (axis cs:-0.1,8.7) {-0.1};

\end{axis}
\end{tikzpicture}

\begin{minipage}{0.85\textwidth}
\small
\textit{Notes:} This figure displays standardized mean differences before (red squares) and after (blue circles) propensity score matching.
The shaded region indicates the $\pm 0.1$ balance threshold recommended by Rosenbaum and Rubin (1985).
Connecting lines show the improvement in balance for each covariate.
All post-matching differences fall within the acceptable range.
\end{minipage}

\end{figure}


% =============================================================================
% Template 4: Propensity Score Distribution
% =============================================================================

\begin{figure}[htbp]
\centering
\caption{Propensity Score Distribution by Treatment Status}
\label{fig:pscore_dist}

\begin{tikzpicture}
\begin{axis}[
    width=0.85\textwidth,
    height=0.5\textwidth,
    xmin=0, xmax=1,
    ymin=0, ymax=5,
    xlabel={Propensity Score},
    ylabel={Density},
    axis lines=left,
    legend style={
        at={(0.98,0.98)},
        anchor=north east,
        draw=none,
        fill=white,
        fill opacity=0.9,
    },
    area style,
    clip=false,
]

% Control group (mirrored below)
\addplot[
    controlgreen,
    fill=controlgreen,
    fill opacity=0.3,
    line width=1pt,
    smooth,
] coordinates {
    (0.00, 0) (0.05, 0.5) (0.10, 1.2) (0.15, 2.0) (0.20, 2.8)
    (0.25, 3.2) (0.30, 3.0) (0.35, 2.5) (0.40, 1.8) (0.45, 1.2)
    (0.50, 0.7) (0.55, 0.4) (0.60, 0.2) (0.65, 0.1) (0.70, 0.05)
    (0.75, 0.02) (0.80, 0.01) (0.85, 0) (1.00, 0)
} \closedcycle;
\addlegendentry{Control}

% Treatment group
\addplot[
    treatred,
    fill=treatred,
    fill opacity=0.3,
    line width=1pt,
    smooth,
] coordinates {
    (0.00, 0) (0.15, 0) (0.20, 0.05) (0.25, 0.2) (0.30, 0.5)
    (0.35, 1.0) (0.40, 1.6) (0.45, 2.2) (0.50, 2.8) (0.55, 3.2)
    (0.60, 3.0) (0.65, 2.5) (0.70, 1.8) (0.75, 1.2) (0.80, 0.6)
    (0.85, 0.3) (0.90, 0.1) (0.95, 0.02) (1.00, 0)
} \closedcycle;
\addlegendentry{Treated}

% Common support region annotation
\draw[black, dashed, line width=1pt] (axis cs:0.20,0) -- (axis cs:0.20,5);
\draw[black, dashed, line width=1pt] (axis cs:0.85,0) -- (axis cs:0.85,5);
\node[anchor=south, font=\small] at (axis cs:0.525,4.5) {Common Support};

\end{axis}
\end{tikzpicture}

\begin{minipage}{0.85\textwidth}
\small
\textit{Notes:} This figure shows the distribution of estimated propensity scores for treated (red) and control (green) groups.
The dashed vertical lines indicate the common support region $[0.20, 0.85]$.
Observations outside this region are trimmed from the matched sample.
The overlap suggests the positivity assumption is reasonably satisfied within the common support.
\end{minipage}

\end{figure}

\end{document}

% =============================================================================
% USAGE NOTES:
% =============================================================================
%
% 1. Balance Diagnostics Thresholds:
%    - Standardized difference: |d| < 0.1 (stringent), |d| < 0.25 (lenient)
%    - Variance ratio: 0.5 < VR < 2 (acceptable range)
%    - Rubin's B < 25, Rubin's R in [0.5, 2]
%
% 2. Common Matching Methods:
%    - Nearest Neighbor: Simple, but may have poor balance
%    - Caliper: Better balance, may lose observations
%    - Kernel: Uses all controls, good for continuous treatment
%    - IPW: No observations lost, sensitive to extreme weights
%
% 3. Reporting Best Practices:
%    - Always show pre-matching imbalance to motivate matching
%    - Report sample sizes before and after matching
%    - Include propensity score model specification in notes
%    - Consider showing propensity score overlap graphically
%
% 4. Sensitivity Analysis:
%    - Vary caliper width (0.01, 0.05, 0.1, 0.25)
%    - Compare matching with replacement vs. without
%    - Test different propensity score models
%    - Report Rosenbaum bounds for hidden bias
%
% 5. Programmatic Generation:
%    - Python: Use causalml or pymatch for balance tables
%    - R: Use MatchIt package's summary() and cobalt::love.plot()
%    - Stata: Use psmatch2 and pstest commands
%
% 6. Publication Tips:
%    - Top journals expect Love plots for visual balance assessment
%    - Report all covariates used in matching, not just selected ones
%    - Include pre-treatment outcome lags if available
%    - Discuss any variables with remaining imbalance
% =============================================================================
