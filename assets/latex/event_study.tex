% =============================================================================
% event_study.tex - Event Study Plot Template using PGFPlots
% =============================================================================
% Style: Publication-quality event study plots for causal inference
% Features:
%   - Pre/post treatment period markers
%   - Confidence intervals as shaded regions
%   - Treatment timing indicator (vertical line)
%   - Reference period normalization
%   - Support for multiple treatment groups
%
% Placeholders (replace with your data):
%   - {{TITLE}}          : Figure title
%   - {{LABEL}}          : LaTeX label for cross-referencing
%   - {{COEF_T*}}        : Coefficient at time T relative to treatment
%   - {{CI_LO_T*}}       : Lower CI bound at time T
%   - {{CI_HI_T*}}       : Upper CI bound at time T
% =============================================================================

\documentclass[12pt]{article}
% =============================================================================
% common_preamble.tex - Shared LaTeX Preamble for Causal Inference Tables/Figures
% =============================================================================
% Usage: % =============================================================================
% common_preamble.tex - Shared LaTeX Preamble for Causal Inference Tables/Figures
% =============================================================================
% Usage: % =============================================================================
% common_preamble.tex - Shared LaTeX Preamble for Causal Inference Tables/Figures
% =============================================================================
% Usage: \input{common_preamble.tex} at the beginning of your document
% Compatible with: pdflatex, xelatex, lualatex
% Style: AER/NBER publication standards
% =============================================================================

% -----------------------------------------------------------------------------
% Required Packages
% -----------------------------------------------------------------------------

% Tables
\usepackage{booktabs}           % Professional table rules (\toprule, \midrule, \bottomrule)
\usepackage{threeparttable}     % Tables with notes
\usepackage{multirow}           % Multi-row cells
\usepackage{array}              % Enhanced column specifications
\usepackage{tabularx}           % Auto-width columns
\usepackage{longtable}          % Multi-page tables

% Number Formatting
\usepackage{siunitx}            % Consistent number formatting and alignment
\sisetup{
    input-symbols = (),         % Allow parentheses in input
    input-open-uncertainty  = , % No uncertainty notation
    input-close-uncertainty = ,
    table-align-text-pre    = false,
    table-align-text-post   = false,
    group-separator         = {,},      % Thousands separator
    group-minimum-digits    = 4,        % Apply grouping for 4+ digits
    round-mode              = places,   % Round to decimal places
    round-precision         = 3,        % Default 3 decimal places
    detect-weight           = true,     % Detect bold
    detect-family           = true,     % Detect font family
}

% Graphics and Plots
\usepackage{graphicx}           % Include graphics
\usepackage{pgfplots}           % Publication-quality plots
\pgfplotsset{compat=1.18}       % Use latest pgfplots features
\usepgfplotslibrary{fillbetween}% For confidence interval shading
\usetikzlibrary{patterns}       % Pattern fills
\usetikzlibrary{arrows.meta}    % Arrow styles

% Colors
\usepackage{xcolor}             % Color support

% Captions
\usepackage{caption}            % Caption customization
\captionsetup{
    font=small,
    labelfont=bf,
    skip=10pt
}

% -----------------------------------------------------------------------------
% Color Definitions (Publication-Friendly Palette)
% -----------------------------------------------------------------------------

% Primary colors for plots
\definecolor{coefblue}{RGB}{31, 119, 180}       % Main coefficient color
\definecolor{ciblue}{RGB}{174, 199, 232}        % Confidence interval fill
\definecolor{treatred}{RGB}{214, 39, 40}        % Treatment/significant
\definecolor{controlgreen}{RGB}{44, 160, 44}    % Control group
\definecolor{refgray}{RGB}{127, 127, 127}       % Reference lines
\definecolor{shadegray}{RGB}{220, 220, 220}     % Shaded regions

% Secondary colors
\definecolor{orange}{RGB}{255, 127, 14}
\definecolor{purple}{RGB}{148, 103, 189}
\definecolor{brown}{RGB}{140, 86, 75}
\definecolor{pink}{RGB}{227, 119, 194}

% -----------------------------------------------------------------------------
% Significance Stars Commands
% -----------------------------------------------------------------------------

% Standard significance stars (AER style)
% Usage: \sym{*}, \sym{**}, \sym{***}
\newcommand{\sym}[1]{\ensuremath{^{#1}}}

% Significance level commands with standard thresholds
% Usage: \stars{0.001} produces ***
\newcommand{\stars}[1]{%
    \ifnum\pdfstrcmp{#1}{0.01}<0 \sym{***}%
    \else\ifnum\pdfstrcmp{#1}{0.05}<0 \sym{**}%
    \else\ifnum\pdfstrcmp{#1}{0.1}<0 \sym{*}%
    \fi\fi\fi%
}

% Pre-defined star commands
\newcommand{\onestar}{\sym{*}}
\newcommand{\twostar}{\sym{**}}
\newcommand{\threestar}{\sym{***}}

% -----------------------------------------------------------------------------
% Table Formatting Commands
% -----------------------------------------------------------------------------

% Standard error in parentheses
% Usage: \se{0.123}
\newcommand{\se}[1]{(#1)}

% Coefficient with standard error
% Usage: \coefse{1.234}{0.123}
\newcommand{\coefse}[2]{#1 \\ \se{#2}}

% Coefficient with SE and stars
% Usage: \coefstars{1.234}{0.123}{**}
\newcommand{\coefstars}[3]{#1\sym{#3} \\ \se{#2}}

% Yes/No indicators for fixed effects
\newcommand{\yes}{Yes}
\newcommand{\no}{No}
\newcommand{\ymark}{\checkmark}

% Panel headers
\newcommand{\panelheader}[1]{\multicolumn{\numcols}{l}{\textit{Panel #1}}}

% -----------------------------------------------------------------------------
% Number Formatting Commands
% -----------------------------------------------------------------------------

% Format large numbers with comma separators
% Usage: \numfmt{12345} produces 12,345
\newcommand{\numfmt}[1]{\num[group-separator={,}]{#1}}

% Format percentages
% Usage: \pct{45.6} produces 45.6%
\newcommand{\pct}[1]{#1\%}

% Format p-values
% Usage: \pval{0.001} produces <0.001 for very small values
\newcommand{\pval}[1]{%
    \ifdim #1 pt < 0.001 pt
        $<$0.001%
    \else
        #1%
    \fi
}

% -----------------------------------------------------------------------------
% Note Formatting for Tables
% -----------------------------------------------------------------------------

% Standard table notes style
\newcommand{\tablenote}[1]{%
    \begin{tablenotes}[flushleft]
        \small
        \item \textit{Notes:} #1
    \end{tablenotes}
}

% Significance note (standard)
\newcommand{\signote}{%
    \sym{*} $p<0.10$, \sym{**} $p<0.05$, \sym{***} $p<0.01$%
}

% Full standard note
\newcommand{\stdnote}[1]{%
    \tablenote{#1 Standard errors in parentheses. \signote.}%
}

% -----------------------------------------------------------------------------
% PGFPlots Style Definitions
% -----------------------------------------------------------------------------

\pgfplotsset{
    % Style for coefficient plots
    coefplot/.style={
        width=0.9\textwidth,
        height=0.6\textwidth,
        xbar,
        bar width=3pt,
        xmin=-2, xmax=2,
        axis lines=left,
        ytick=data,
        enlarge y limits=0.15,
        xlabel={Coefficient Estimate},
        every axis x label/.style={
            at={(ticklabel* cs:1.02)},
            anchor=west,
        },
        x axis line style={-},
        y axis line style={opacity=0},
        tickwidth=0pt,
        clip=false,
    },
    % Style for event study plots
    eventstudy/.style={
        width=0.9\textwidth,
        height=0.5\textwidth,
        axis lines=middle,
        xlabel={Time Relative to Treatment},
        ylabel={Coefficient},
        xmin=-6, xmax=6,
        xtick={-5,-4,-3,-2,-1,0,1,2,3,4,5},
        extra x ticks={0},
        extra x tick style={grid=major, grid style={dashed, red!50}},
        legend pos=north west,
        legend style={draw=none, fill=none},
    },
    % Common style elements
    every axis/.append style={
        font=\small,
        tick label style={font=\small},
        label style={font=\small},
        legend style={font=\small},
    },
}

% -----------------------------------------------------------------------------
% End of Preamble
% -----------------------------------------------------------------------------
 at the beginning of your document
% Compatible with: pdflatex, xelatex, lualatex
% Style: AER/NBER publication standards
% =============================================================================

% -----------------------------------------------------------------------------
% Required Packages
% -----------------------------------------------------------------------------

% Tables
\usepackage{booktabs}           % Professional table rules (\toprule, \midrule, \bottomrule)
\usepackage{threeparttable}     % Tables with notes
\usepackage{multirow}           % Multi-row cells
\usepackage{array}              % Enhanced column specifications
\usepackage{tabularx}           % Auto-width columns
\usepackage{longtable}          % Multi-page tables

% Number Formatting
\usepackage{siunitx}            % Consistent number formatting and alignment
\sisetup{
    input-symbols = (),         % Allow parentheses in input
    input-open-uncertainty  = , % No uncertainty notation
    input-close-uncertainty = ,
    table-align-text-pre    = false,
    table-align-text-post   = false,
    group-separator         = {,},      % Thousands separator
    group-minimum-digits    = 4,        % Apply grouping for 4+ digits
    round-mode              = places,   % Round to decimal places
    round-precision         = 3,        % Default 3 decimal places
    detect-weight           = true,     % Detect bold
    detect-family           = true,     % Detect font family
}

% Graphics and Plots
\usepackage{graphicx}           % Include graphics
\usepackage{pgfplots}           % Publication-quality plots
\pgfplotsset{compat=1.18}       % Use latest pgfplots features
\usepgfplotslibrary{fillbetween}% For confidence interval shading
\usetikzlibrary{patterns}       % Pattern fills
\usetikzlibrary{arrows.meta}    % Arrow styles

% Colors
\usepackage{xcolor}             % Color support

% Captions
\usepackage{caption}            % Caption customization
\captionsetup{
    font=small,
    labelfont=bf,
    skip=10pt
}

% -----------------------------------------------------------------------------
% Color Definitions (Publication-Friendly Palette)
% -----------------------------------------------------------------------------

% Primary colors for plots
\definecolor{coefblue}{RGB}{31, 119, 180}       % Main coefficient color
\definecolor{ciblue}{RGB}{174, 199, 232}        % Confidence interval fill
\definecolor{treatred}{RGB}{214, 39, 40}        % Treatment/significant
\definecolor{controlgreen}{RGB}{44, 160, 44}    % Control group
\definecolor{refgray}{RGB}{127, 127, 127}       % Reference lines
\definecolor{shadegray}{RGB}{220, 220, 220}     % Shaded regions

% Secondary colors
\definecolor{orange}{RGB}{255, 127, 14}
\definecolor{purple}{RGB}{148, 103, 189}
\definecolor{brown}{RGB}{140, 86, 75}
\definecolor{pink}{RGB}{227, 119, 194}

% -----------------------------------------------------------------------------
% Significance Stars Commands
% -----------------------------------------------------------------------------

% Standard significance stars (AER style)
% Usage: \sym{*}, \sym{**}, \sym{***}
\newcommand{\sym}[1]{\ensuremath{^{#1}}}

% Significance level commands with standard thresholds
% Usage: \stars{0.001} produces ***
\newcommand{\stars}[1]{%
    \ifnum\pdfstrcmp{#1}{0.01}<0 \sym{***}%
    \else\ifnum\pdfstrcmp{#1}{0.05}<0 \sym{**}%
    \else\ifnum\pdfstrcmp{#1}{0.1}<0 \sym{*}%
    \fi\fi\fi%
}

% Pre-defined star commands
\newcommand{\onestar}{\sym{*}}
\newcommand{\twostar}{\sym{**}}
\newcommand{\threestar}{\sym{***}}

% -----------------------------------------------------------------------------
% Table Formatting Commands
% -----------------------------------------------------------------------------

% Standard error in parentheses
% Usage: \se{0.123}
\newcommand{\se}[1]{(#1)}

% Coefficient with standard error
% Usage: \coefse{1.234}{0.123}
\newcommand{\coefse}[2]{#1 \\ \se{#2}}

% Coefficient with SE and stars
% Usage: \coefstars{1.234}{0.123}{**}
\newcommand{\coefstars}[3]{#1\sym{#3} \\ \se{#2}}

% Yes/No indicators for fixed effects
\newcommand{\yes}{Yes}
\newcommand{\no}{No}
\newcommand{\ymark}{\checkmark}

% Panel headers
\newcommand{\panelheader}[1]{\multicolumn{\numcols}{l}{\textit{Panel #1}}}

% -----------------------------------------------------------------------------
% Number Formatting Commands
% -----------------------------------------------------------------------------

% Format large numbers with comma separators
% Usage: \numfmt{12345} produces 12,345
\newcommand{\numfmt}[1]{\num[group-separator={,}]{#1}}

% Format percentages
% Usage: \pct{45.6} produces 45.6%
\newcommand{\pct}[1]{#1\%}

% Format p-values
% Usage: \pval{0.001} produces <0.001 for very small values
\newcommand{\pval}[1]{%
    \ifdim #1 pt < 0.001 pt
        $<$0.001%
    \else
        #1%
    \fi
}

% -----------------------------------------------------------------------------
% Note Formatting for Tables
% -----------------------------------------------------------------------------

% Standard table notes style
\newcommand{\tablenote}[1]{%
    \begin{tablenotes}[flushleft]
        \small
        \item \textit{Notes:} #1
    \end{tablenotes}
}

% Significance note (standard)
\newcommand{\signote}{%
    \sym{*} $p<0.10$, \sym{**} $p<0.05$, \sym{***} $p<0.01$%
}

% Full standard note
\newcommand{\stdnote}[1]{%
    \tablenote{#1 Standard errors in parentheses. \signote.}%
}

% -----------------------------------------------------------------------------
% PGFPlots Style Definitions
% -----------------------------------------------------------------------------

\pgfplotsset{
    % Style for coefficient plots
    coefplot/.style={
        width=0.9\textwidth,
        height=0.6\textwidth,
        xbar,
        bar width=3pt,
        xmin=-2, xmax=2,
        axis lines=left,
        ytick=data,
        enlarge y limits=0.15,
        xlabel={Coefficient Estimate},
        every axis x label/.style={
            at={(ticklabel* cs:1.02)},
            anchor=west,
        },
        x axis line style={-},
        y axis line style={opacity=0},
        tickwidth=0pt,
        clip=false,
    },
    % Style for event study plots
    eventstudy/.style={
        width=0.9\textwidth,
        height=0.5\textwidth,
        axis lines=middle,
        xlabel={Time Relative to Treatment},
        ylabel={Coefficient},
        xmin=-6, xmax=6,
        xtick={-5,-4,-3,-2,-1,0,1,2,3,4,5},
        extra x ticks={0},
        extra x tick style={grid=major, grid style={dashed, red!50}},
        legend pos=north west,
        legend style={draw=none, fill=none},
    },
    % Common style elements
    every axis/.append style={
        font=\small,
        tick label style={font=\small},
        label style={font=\small},
        legend style={font=\small},
    },
}

% -----------------------------------------------------------------------------
% End of Preamble
% -----------------------------------------------------------------------------
 at the beginning of your document
% Compatible with: pdflatex, xelatex, lualatex
% Style: AER/NBER publication standards
% =============================================================================

% -----------------------------------------------------------------------------
% Required Packages
% -----------------------------------------------------------------------------

% Tables
\usepackage{booktabs}           % Professional table rules (\toprule, \midrule, \bottomrule)
\usepackage{threeparttable}     % Tables with notes
\usepackage{multirow}           % Multi-row cells
\usepackage{array}              % Enhanced column specifications
\usepackage{tabularx}           % Auto-width columns
\usepackage{longtable}          % Multi-page tables

% Number Formatting
\usepackage{siunitx}            % Consistent number formatting and alignment
\sisetup{
    input-symbols = (),         % Allow parentheses in input
    input-open-uncertainty  = , % No uncertainty notation
    input-close-uncertainty = ,
    table-align-text-pre    = false,
    table-align-text-post   = false,
    group-separator         = {,},      % Thousands separator
    group-minimum-digits    = 4,        % Apply grouping for 4+ digits
    round-mode              = places,   % Round to decimal places
    round-precision         = 3,        % Default 3 decimal places
    detect-weight           = true,     % Detect bold
    detect-family           = true,     % Detect font family
}

% Graphics and Plots
\usepackage{graphicx}           % Include graphics
\usepackage{pgfplots}           % Publication-quality plots
\pgfplotsset{compat=1.18}       % Use latest pgfplots features
\usepgfplotslibrary{fillbetween}% For confidence interval shading
\usetikzlibrary{patterns}       % Pattern fills
\usetikzlibrary{arrows.meta}    % Arrow styles

% Colors
\usepackage{xcolor}             % Color support

% Captions
\usepackage{caption}            % Caption customization
\captionsetup{
    font=small,
    labelfont=bf,
    skip=10pt
}

% -----------------------------------------------------------------------------
% Color Definitions (Publication-Friendly Palette)
% -----------------------------------------------------------------------------

% Primary colors for plots
\definecolor{coefblue}{RGB}{31, 119, 180}       % Main coefficient color
\definecolor{ciblue}{RGB}{174, 199, 232}        % Confidence interval fill
\definecolor{treatred}{RGB}{214, 39, 40}        % Treatment/significant
\definecolor{controlgreen}{RGB}{44, 160, 44}    % Control group
\definecolor{refgray}{RGB}{127, 127, 127}       % Reference lines
\definecolor{shadegray}{RGB}{220, 220, 220}     % Shaded regions

% Secondary colors
\definecolor{orange}{RGB}{255, 127, 14}
\definecolor{purple}{RGB}{148, 103, 189}
\definecolor{brown}{RGB}{140, 86, 75}
\definecolor{pink}{RGB}{227, 119, 194}

% -----------------------------------------------------------------------------
% Significance Stars Commands
% -----------------------------------------------------------------------------

% Standard significance stars (AER style)
% Usage: \sym{*}, \sym{**}, \sym{***}
\newcommand{\sym}[1]{\ensuremath{^{#1}}}

% Significance level commands with standard thresholds
% Usage: \stars{0.001} produces ***
\newcommand{\stars}[1]{%
    \ifnum\pdfstrcmp{#1}{0.01}<0 \sym{***}%
    \else\ifnum\pdfstrcmp{#1}{0.05}<0 \sym{**}%
    \else\ifnum\pdfstrcmp{#1}{0.1}<0 \sym{*}%
    \fi\fi\fi%
}

% Pre-defined star commands
\newcommand{\onestar}{\sym{*}}
\newcommand{\twostar}{\sym{**}}
\newcommand{\threestar}{\sym{***}}

% -----------------------------------------------------------------------------
% Table Formatting Commands
% -----------------------------------------------------------------------------

% Standard error in parentheses
% Usage: \se{0.123}
\newcommand{\se}[1]{(#1)}

% Coefficient with standard error
% Usage: \coefse{1.234}{0.123}
\newcommand{\coefse}[2]{#1 \\ \se{#2}}

% Coefficient with SE and stars
% Usage: \coefstars{1.234}{0.123}{**}
\newcommand{\coefstars}[3]{#1\sym{#3} \\ \se{#2}}

% Yes/No indicators for fixed effects
\newcommand{\yes}{Yes}
\newcommand{\no}{No}
\newcommand{\ymark}{\checkmark}

% Panel headers
\newcommand{\panelheader}[1]{\multicolumn{\numcols}{l}{\textit{Panel #1}}}

% -----------------------------------------------------------------------------
% Number Formatting Commands
% -----------------------------------------------------------------------------

% Format large numbers with comma separators
% Usage: \numfmt{12345} produces 12,345
\newcommand{\numfmt}[1]{\num[group-separator={,}]{#1}}

% Format percentages
% Usage: \pct{45.6} produces 45.6%
\newcommand{\pct}[1]{#1\%}

% Format p-values
% Usage: \pval{0.001} produces <0.001 for very small values
\newcommand{\pval}[1]{%
    \ifdim #1 pt < 0.001 pt
        $<$0.001%
    \else
        #1%
    \fi
}

% -----------------------------------------------------------------------------
% Note Formatting for Tables
% -----------------------------------------------------------------------------

% Standard table notes style
\newcommand{\tablenote}[1]{%
    \begin{tablenotes}[flushleft]
        \small
        \item \textit{Notes:} #1
    \end{tablenotes}
}

% Significance note (standard)
\newcommand{\signote}{%
    \sym{*} $p<0.10$, \sym{**} $p<0.05$, \sym{***} $p<0.01$%
}

% Full standard note
\newcommand{\stdnote}[1]{%
    \tablenote{#1 Standard errors in parentheses. \signote.}%
}

% -----------------------------------------------------------------------------
% PGFPlots Style Definitions
% -----------------------------------------------------------------------------

\pgfplotsset{
    % Style for coefficient plots
    coefplot/.style={
        width=0.9\textwidth,
        height=0.6\textwidth,
        xbar,
        bar width=3pt,
        xmin=-2, xmax=2,
        axis lines=left,
        ytick=data,
        enlarge y limits=0.15,
        xlabel={Coefficient Estimate},
        every axis x label/.style={
            at={(ticklabel* cs:1.02)},
            anchor=west,
        },
        x axis line style={-},
        y axis line style={opacity=0},
        tickwidth=0pt,
        clip=false,
    },
    % Style for event study plots
    eventstudy/.style={
        width=0.9\textwidth,
        height=0.5\textwidth,
        axis lines=middle,
        xlabel={Time Relative to Treatment},
        ylabel={Coefficient},
        xmin=-6, xmax=6,
        xtick={-5,-4,-3,-2,-1,0,1,2,3,4,5},
        extra x ticks={0},
        extra x tick style={grid=major, grid style={dashed, red!50}},
        legend pos=north west,
        legend style={draw=none, fill=none},
    },
    % Common style elements
    every axis/.append style={
        font=\small,
        tick label style={font=\small},
        label style={font=\small},
        legend style={font=\small},
    },
}

% -----------------------------------------------------------------------------
% End of Preamble
% -----------------------------------------------------------------------------


\begin{document}

% =============================================================================
% Template 1: Basic Event Study with Shaded Confidence Intervals
% =============================================================================

\begin{figure}[htbp]
\centering
\caption{{{TITLE}}: Event Study Estimates}
\label{fig:{{LABEL}}_eventstudy}

\begin{tikzpicture}
\begin{axis}[
    % Layout
    width=0.9\textwidth,
    height=0.55\textwidth,
    % Axis range
    xmin=-6.5, xmax=6.5,
    ymin=-0.8, ymax=1.2,
    % Labels
    xlabel={Time Relative to Treatment ($t=0$)},
    ylabel={Coefficient Estimate},
    % Ticks
    xtick={-6,-5,-4,-3,-2,-1,0,1,2,3,4,5,6},
    xticklabels={$-6$,$-5$,$-4$,$-3$,$-2$,$-1$,$0$,$+1$,$+2$,$+3$,$+4$,$+5$,$+6$},
    ytick={-0.6,-0.4,-0.2,0,0.2,0.4,0.6,0.8,1.0},
    minor y tick num=1,
    % Axis styling
    axis lines=middle,
    axis line style={-},
    tick align=outside,
    % Grid
    ymajorgrids=true,
    grid style={dashed, gray!30},
    % Reference line at y=0
    extra y ticks={0},
    extra y tick style={
        grid=major,
        grid style={solid, gray!50, line width=0.5pt},
        tick label style={opacity=0}
    },
    % Legend
    legend pos=north west,
    legend style={
        draw=none,
        fill=white,
        fill opacity=0.9,
        text opacity=1,
        font=\small,
    },
    clip=false,
]

% ----- Treatment timing indicator (vertical line at t=0) -----
\draw[treatred, line width=1.5pt, dashed]
    (axis cs:0,\pgfkeysvalueof{/pgfplots/ymin}) --
    (axis cs:0,\pgfkeysvalueof{/pgfplots/ymax});

% ----- Confidence interval shaded region -----
% Define upper and lower bounds
\addplot[
    name path=upper,
    draw=none,
    forget plot,
] coordinates {
    (-6, {{CI_HI_T-6}})
    (-5, {{CI_HI_T-5}})
    (-4, {{CI_HI_T-4}})
    (-3, {{CI_HI_T-3}})
    (-2, {{CI_HI_T-2}})
    (-1, 0)              % Reference period (normalized to 0)
    (0, {{CI_HI_T0}})
    (1, {{CI_HI_T1}})
    (2, {{CI_HI_T2}})
    (3, {{CI_HI_T3}})
    (4, {{CI_HI_T4}})
    (5, {{CI_HI_T5}})
    (6, {{CI_HI_T6}})
};

\addplot[
    name path=lower,
    draw=none,
    forget plot,
] coordinates {
    (-6, {{CI_LO_T-6}})
    (-5, {{CI_LO_T-5}})
    (-4, {{CI_LO_T-4}})
    (-3, {{CI_LO_T-3}})
    (-2, {{CI_LO_T-2}})
    (-1, 0)              % Reference period
    (0, {{CI_LO_T0}})
    (1, {{CI_LO_T1}})
    (2, {{CI_LO_T2}})
    (3, {{CI_LO_T3}})
    (4, {{CI_LO_T4}})
    (5, {{CI_LO_T5}})
    (6, {{CI_LO_T6}})
};

% Fill between upper and lower
\addplot[
    fill=ciblue,
    fill opacity=0.3,
    draw=none,
] fill between[of=upper and lower];

% ----- Point estimates with markers -----
\addplot[
    coefblue,
    line width=1.5pt,
    mark=*,
    mark size=3pt,
    mark options={solid, fill=coefblue},
] coordinates {
    (-6, {{COEF_T-6}})
    (-5, {{COEF_T-5}})
    (-4, {{COEF_T-4}})
    (-3, {{COEF_T-3}})
    (-2, {{COEF_T-2}})
    (-1, 0)              % Reference period (normalized to 0)
    (0, {{COEF_T0}})
    (1, {{COEF_T1}})
    (2, {{COEF_T2}})
    (3, {{COEF_T3}})
    (4, {{COEF_T4}})
    (5, {{COEF_T5}})
    (6, {{COEF_T6}})
};
\addlegendentry{Point Estimate}

% ----- Reference period marker -----
\addplot[
    only marks,
    mark=o,
    mark size=5pt,
    mark options={solid, draw=black, line width=1.5pt, fill=white},
    forget plot,
] coordinates {(-1, 0)};

% ----- Annotation for reference period -----
\node[anchor=south, font=\footnotesize] at (axis cs:-1,0.1) {Ref.};

% ----- Pre-treatment region label -----
\node[anchor=south, font=\small, gray] at (axis cs:-3.5,1.05) {Pre-Treatment};
\node[anchor=south, font=\small, gray] at (axis cs:3,1.05) {Post-Treatment};

\end{axis}
\end{tikzpicture}

\begin{minipage}{0.9\textwidth}
\small
\textit{Notes:} This figure presents event study estimates of the treatment effect.
The x-axis shows time relative to treatment, with $t=0$ being the treatment period.
The omitted reference period is $t=-1$.
Point estimates (circles) are connected by a solid line; the shaded region shows 95\% confidence intervals.
The dashed vertical line indicates treatment timing.
Standard errors are clustered at the [UNIT] level.
\end{minipage}

\end{figure}


% =============================================================================
% Template 2: Event Study with Error Bars (Alternative Style)
% =============================================================================

\begin{figure}[htbp]
\centering
\caption{Event Study: Error Bar Style}
\label{fig:eventstudy_errorbars}

\begin{tikzpicture}
\begin{axis}[
    width=0.9\textwidth,
    height=0.5\textwidth,
    xmin=-5.5, xmax=5.5,
    ymin=-0.6, ymax=0.8,
    xlabel={Periods Relative to Treatment},
    ylabel={Coefficient},
    xtick={-5,-4,-3,-2,-1,0,1,2,3,4,5},
    axis lines=middle,
    ymajorgrids=true,
    grid style={dashed, gray!30},
    clip=false,
]

% Treatment timing line
\draw[treatred, line width=1.5pt, dashed]
    (axis cs:-0.5,\pgfkeysvalueof{/pgfplots/ymin}) --
    (axis cs:-0.5,\pgfkeysvalueof{/pgfplots/ymax});

% Point estimates with error bars
\addplot[
    coefblue,
    only marks,
    mark=*,
    mark size=3pt,
    error bars/.cd,
    y dir=both,
    y explicit,
    error bar style={line width=1pt, coefblue},
    error mark options={mark size=3pt, line width=1pt, coefblue}
] coordinates {
    (-5, 0.02) +- (0, 0.15)
    (-4, -0.05) +- (0, 0.12)
    (-3, 0.03) +- (0, 0.11)
    (-2, 0.01) +- (0, 0.10)
    (-1, 0) +- (0, 0)           % Reference (no error bar)
    (0, 0.25) +- (0, 0.13)
    (1, 0.42) +- (0, 0.14)
    (2, 0.48) +- (0, 0.15)
    (3, 0.51) +- (0, 0.16)
    (4, 0.53) +- (0, 0.18)
    (5, 0.50) +- (0, 0.20)
};

% Horizontal reference line at y=0
\draw[gray, line width=0.5pt] (axis cs:-5.5,0) -- (axis cs:5.5,0);

% Reference period marker
\addplot[
    only marks,
    mark=o,
    mark size=5pt,
    mark options={solid, draw=black, line width=1.5pt, fill=white},
] coordinates {(-1, 0)};

\end{axis}
\end{tikzpicture}

\end{figure}


% =============================================================================
% Template 3: Multiple Groups Event Study
% =============================================================================

\begin{figure}[htbp]
\centering
\caption{Event Study: Heterogeneous Effects by Group}
\label{fig:eventstudy_hetero}

\begin{tikzpicture}
\begin{axis}[
    width=0.9\textwidth,
    height=0.55\textwidth,
    xmin=-5.5, xmax=5.5,
    ymin=-0.4, ymax=1.0,
    xlabel={Periods Relative to Treatment},
    ylabel={Coefficient Estimate},
    xtick={-5,-4,-3,-2,-1,0,1,2,3,4,5},
    axis lines=middle,
    ymajorgrids=true,
    grid style={dashed, gray!30},
    legend style={
        at={(0.02,0.98)},
        anchor=north west,
        draw=none,
        fill=white,
        fill opacity=0.9,
    },
    clip=false,
]

% Treatment timing
\draw[gray, line width=1pt, dashed]
    (axis cs:-0.5,-0.4) -- (axis cs:-0.5,1.0);

% Group 1: High Intensity
\addplot[
    treatred,
    line width=1.5pt,
    mark=square*,
    mark size=2.5pt,
] coordinates {
    (-5, 0.03) (-4, -0.02) (-3, 0.01) (-2, 0.02) (-1, 0)
    (0, 0.35) (1, 0.58) (2, 0.72) (3, 0.78) (4, 0.82) (5, 0.80)
};
\addlegendentry{High Intensity}

% Group 2: Medium Intensity
\addplot[
    coefblue,
    line width=1.5pt,
    mark=*,
    mark size=2.5pt,
] coordinates {
    (-5, 0.01) (-4, 0.02) (-3, -0.01) (-2, 0.01) (-1, 0)
    (0, 0.20) (1, 0.35) (2, 0.42) (3, 0.45) (4, 0.47) (5, 0.46)
};
\addlegendentry{Medium Intensity}

% Group 3: Low Intensity
\addplot[
    controlgreen,
    line width=1.5pt,
    mark=triangle*,
    mark size=3pt,
] coordinates {
    (-5, -0.02) (-4, 0.01) (-3, 0.02) (-2, -0.01) (-1, 0)
    (0, 0.08) (1, 0.15) (2, 0.18) (3, 0.19) (4, 0.20) (5, 0.19)
};
\addlegendentry{Low Intensity}

% Reference line at y=0
\draw[gray, line width=0.5pt] (axis cs:-5.5,0) -- (axis cs:5.5,0);

\end{axis}
\end{tikzpicture}

\begin{minipage}{0.9\textwidth}
\small
\textit{Notes:} This figure presents event study estimates by treatment intensity.
High intensity (red squares), medium intensity (blue circles), and low intensity (green triangles) groups are shown separately.
The omitted reference period is $t=-1$. All groups show parallel pre-trends and divergent post-treatment effects.
\end{minipage}

\end{figure}


% =============================================================================
% Template 4: Stacked/Panel Event Study
% =============================================================================

\begin{figure}[htbp]
\centering
\caption{Event Study: Multiple Outcomes}
\label{fig:eventstudy_panel}

\begin{tikzpicture}
% Panel A
\begin{axis}[
    name=panelA,
    width=0.45\textwidth,
    height=0.35\textwidth,
    xmin=-4.5, xmax=4.5,
    ymin=-0.3, ymax=0.6,
    xlabel={},
    ylabel={Outcome A},
    xtick={-4,-3,-2,-1,0,1,2,3,4},
    title={\textbf{Panel A: Primary Outcome}},
    title style={at={(0.5,1.05)}, anchor=south},
    ymajorgrids=true,
    grid style={dashed, gray!20},
]
\draw[treatred, dashed] (axis cs:-0.5,-0.3) -- (axis cs:-0.5,0.6);
\addplot[coefblue, line width=1pt, mark=*, mark size=2pt] coordinates {
    (-4, 0.01) (-3, -0.02) (-2, 0.01) (-1, 0)
    (0, 0.15) (1, 0.28) (2, 0.35) (3, 0.38) (4, 0.40)
};
\end{axis}

% Panel B
\begin{axis}[
    name=panelB,
    at={($(panelA.east)+(1.2cm,0)$)},
    anchor=west,
    width=0.45\textwidth,
    height=0.35\textwidth,
    xmin=-4.5, xmax=4.5,
    ymin=-0.3, ymax=0.6,
    xlabel={},
    ylabel={Outcome B},
    xtick={-4,-3,-2,-1,0,1,2,3,4},
    title={\textbf{Panel B: Secondary Outcome}},
    title style={at={(0.5,1.05)}, anchor=south},
    ymajorgrids=true,
    grid style={dashed, gray!20},
]
\draw[treatred, dashed] (axis cs:-0.5,-0.3) -- (axis cs:-0.5,0.6);
\addplot[controlgreen, line width=1pt, mark=square*, mark size=2pt] coordinates {
    (-4, -0.01) (-3, 0.02) (-2, -0.01) (-1, 0)
    (0, 0.08) (1, 0.12) (2, 0.15) (3, 0.14) (4, 0.13)
};
\end{axis}

% Common x-axis label
\node[below=0.8cm] at ($(panelA.south)!0.5!(panelB.south)$) {Periods Relative to Treatment};

\end{tikzpicture}

\end{figure}

\end{document}

% =============================================================================
% USAGE NOTES:
% =============================================================================
%
% 1. Reference Period Selection:
%    - Standard: t=-1 (last pre-treatment period)
%    - Alternative: t=-2 for robustness
%    - The reference period coefficient is normalized to 0 by construction
%
% 2. Pre-Trends Testing:
%    - Visual: Pre-treatment coefficients should be close to zero
%    - Statistical: Report joint F-test for pre-treatment coefficients
%    - Add horizontal band at y=0 for visual reference
%
% 3. Confidence Interval Styles:
%    - Shaded region: Best for single series, shows uncertainty well
%    - Error bars: Better for multiple series comparison
%    - Both: Use error bars with light shading for emphasis
%
% 4. Treatment Timing:
%    - Dashed line at t=0 or t=-0.5 (between periods)
%    - Use treatred color to highlight treatment onset
%    - Add arrow or annotation if treatment is gradual
%
% 5. Data Input for Programmatic Use:
%    Option A: Replace {{PLACEHOLDER}} with values
%    Option B: Load from file:
%      \pgfplotstableread[col sep=comma]{eventstudy_data.csv}\esdata
%      \addplot table[x=t, y=coef]{\esdata};
%
% 6. Common Modifications:
%    - Long pre-period: Extend xmin, add more coordinates
%    - Staggered treatment: Use callaway-santanna or sun-abraham estimators
%    - Binary outcome: Consider log-odds scale for y-axis
%
% 7. Publication Tips:
%    - Always include notes explaining the reference period
%    - Report standard error type (clustered, robust, etc.)
%    - Consider adding parallel trends test results in notes
%    - For working papers: include full regression tables in appendix
% =============================================================================
